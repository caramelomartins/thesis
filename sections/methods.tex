\chapter{Methodology}

In previous sections (Chapter \ref{chap:intro} and Chapter \ref{chap:related}) we have motivated our thesis, by describing the current state-of-art, defining and motivating a specific research problem, setting out the objectives and describing the contributions made by this thesis. It is now relevant to explain what methodology was used in our research and the factors that contributed to choosing a given methodology. We will also describe how the various sections of this thesis intersect and fit into the chosen methodology.

\section{Motivation}

\gls{is} has devoted considerable effort to studying methods and frameworks for IS research. Currently, two main paradigms characterize IS research: \gls{bs} and \gls{ds} \cite[76]{Hevner:2004:DSI:2017212.2017217}. After carefully evaluating the existing reference literature, such as \cite{Hevner:2004:DSI:2017212.2017217}, \cite{march1995design}, \cite{winter2008design} or \cite{10.2307/40398896}, we decided that a suitable research methodology would be to conduct our research based on the guidelines for \gls{dsr}, as it is a common and accepted practice in \gls{is} research. We have also integrated a more comprehensive framework for the evaluation section of \gls{dsr} research \cite{10.1007/978-3-642-29863-9_31} developed by \citeauthor{10.1007/978-3-642-29863-9_31}, based on \cite{pries2008strategies}.

In their seminal paper, \citeauthor{march1995design} define a \gls{ds} research framework by applying natural science methodologies to the study of \gls{is}, with the purpose of maximizing research utility, contrasting with the natural sciences' view of research for truth (\cite[80]{Hevner:2004:DSI:2017212.2017217}, \cite[253]{march1995design}). They achieve that by creating a framework composed by a two-dimensional matrix: research outputs and research activities \cite[255]{march1995design}. Research outputs are composed by \textit{constructs}, \textit{models}, \textit{methods} and \textit{instantiations}. Research activities, on the other hand, are composed by processes such as \textit{build} and \textit{evaluate}. Research outputs, also known as \textit{artifacts}, are created with the purpose of solving a specific problem that has yet to be solved \cite[78]{Hevner:2004:DSI:2017212.2017217}, while research activities are the processes by which those are artifacts are designed to meet a given criteria \cite[79--80]{Hevner:2004:DSI:2017212.2017217}. \citeauthor{Hevner:2004:DSI:2017212.2017217} argued that, in fact, truth and utility are inseparable \cite[80]{Hevner:2004:DSI:2017212.2017217}. This statement is the basis for their approach, which extends the original \gls{dsr} methodology presented by \citeauthor{march1995design}, by setting the following practical guidelines: \emph{(i)} Design as an Artifact, \emph{(ii)} Problem Relevance, \emph{(iii)} Design Evaluation, \emph{(iv)} Research Contribution, \emph{(v)} Research Rigor, \emph{(vi)} Design as a Search Process and \emph{(vii)} Communication of Research. Summarizing, \gls{dsr} requires the creation of an novel, innovative, purposeful artifact, that is formally defined, for application on a specific problem domain, created through a search process to find an effective solution and communicated effectively to the community \cite[82]{Hevner:2004:DSI:2017212.2017217}.

Research presented in \cite{march1995design} and \cite{Hevner:2004:DSI:2017212.2017217} has been extended through research proposed in \cite{10.2307/40398896}. In \cite{10.2307/40398896}, \citeauthor{10.1007/978-3-642-29863-9_31} present a \gls{dsrm} "for the production and presentation of \gls{ds} research in \gls{is}" \cite[3]{10.2307/40398896}. \citeauthor{10.2307/40398896} expanded on previous concepts by proposing a missing procedure \cite[7]{10.2307/40398896} that would connect both the principles and practical guidelines existing in \gls{ds} to create a formal research methodology. With that in mind, they have proposed a series of activities that form a process by which to conduct \gls{dsr}: \emph{(i)} Problem Identification and Motivation, \emph{(ii)} Define the Objectives for a Solution, \emph{(iii)} Design and Development, \emph{(iv)} Demonstration, \emph{(v)} Evaluation and \emph{(vi)} Communication. This set of activities align well with the practical rules proposed by \citeauthor{Hevner:2004:DSI:2017212.2017217} and, as \citeauthor{10.2307/40398896} argue, with most of the literature conducted thus far. Although the process is presented sequentially, the authors point out that "there is no expectation that researchers would always proceed in sequential order from activity one through activity six" \cite[14]{10.2307/40398896}, which provides flexibility while conducting research while, at the same time, providing a rigorous process to follow. The importance of this work is that it takes the practical guidelines and principles established previously and turned them into a process for application on research.

We have aimed, with the research presented in this thesis, to follow what has been proposed by \citeauthor{10.2307/40398896} while keeping in mind the practical guidelines laid out by \citeauthor{Hevner:2004:DSI:2017212.2017217}. Both papers are widely cited contributions to the body of knowledge, with case studies that exemplify its applicability to conducting \gls{dsr}. They have matured through years of refinement and have been widly use by the \gls{is} research community, making them a suitable choice for a methodology to follow.

\section{Application}

To demonstrate the application of the chosen methodology throughout this thesis, we will take the same approach as \cite{10.2307/40398896} by describing how we have fulfilled each of the designated activities.

\subsection{Problem Identification and Motivation}

In Chapter \ref{chap:intro}, we have introduced our thesis by motivating the existence of a specific problem and describing, and justifying, why a solution to this problem is needed. We have exposed a knowledge gap and the relevance of this problem. We justified the solution to the problem by using examples from the past and by connecting this problem with future trends, showing how the problem will be aggravated by the trends in \gls{is} we have been witnessing. We have taken this a step further, by dissecting in Chapter \ref{chap:related} where that knowledge gap is in the middle of the existing literature. This approach also relates correctly with what has been proposed \citeauthor{Hevner:2004:DSI:2017212.2017217} by justifying the \textit{Problem Relevance} of this type of research.

\subsection{Define the Objectives for a Solution}

Again, this has been done in Chapter \ref{chap:intro}, where we have discussed the objectives for this thesis. In this case, we have defined what a solution should look like in order to be an answer to our research question and identifier problem. According to what has been proposed by \citeauthor{10.2307/40398896}, we have proposed qualitative objectives, in the sense that we have provided a description of new artifacts and how they should solve the problem that has been identified. By now, we are already abiding by the \textit{Design as an Artifact} guideline \cite{Hevner:2004:DSI:2017212.2017217} due to the fact that we are defining artifacts as a solution to a given problem.

\subsection{Design and Development}

This component of the process can be found throughout Chapter \ref{chap:study} and \ref{chap:implementation}. We have initially designed high-level artifacts, as models and methods, of what a possible solution to the problem would look like. We conducted a study, in order to assess what would be the perceptions of users to our new artifacts. This study is also a property of informational resource, as defined by \citeauthor{10.2307/40398896}. After that, we took the feedback into consideration, closing the loop, by designing lower-level artifacts, in the form of models, which we used to define methods and then instantiations. These are the artifacts created during our research, following this methodology, in order to evolve from the stage of objectives to creating an actual artifact that represent a solution.

\subsection{Demonstration}

\subsection{Evaluation}

\subsection{Communication}

