\chapter{Methodology}
\label{chap:methods}

In previous sections (Chapter \ref{chap:intro} and Chapter \ref{chap:related}) we motivate our thesis, by analyzing the existing literature, defining and motivating a specific research problem, setting out the objectives and describing the contributions made by this research. It is now relevant to explain the methodology used in our research and the factors that contributed to choosing it. We will also describe how the various sections of this thesis intersect and fit into the chosen methodology, through the adoption of the same presentation model described by the methodology's authors.

\gls{is} has devoted considerable effort to studying methods and frameworks for IS research. Currently, two main paradigms characterize IS research: \gls{bs} and \gls{ds} \cite[76]{hevner_design_2004}. After carefully evaluating existing reference literature, such as \cite{hevner_design_2004}, \cite{march_design_1995}, \cite{winter_design_2008} or \cite{peffers_design_2007}, a suitable research methodology would be to conduct our research based on the guidelines for \gls{dsr}, as it is a common and accepted practice in \gls{is} research.

In their seminal paper, \citeauthor{march_design_1995} \cite{march_design_1995} define a \gls{ds} research framework by applying natural science methodologies to the study of \gls{is}, with the purpose of maximizing research utility, contrasting with the natural sciences' view of research for truth (\cite[80]{hevner_design_2004}, \cite[253]{march_design_1995}). They achieve that by creating a framework consisting of a two-dimensional matrix: research outputs and research activities \cite[255]{march_design_1995}. Research outputs are composed of \textit{constructs}, \textit{models}, \textit{methods} and \textit{instantiations}. Research activities are composed of processes such as \textit{build} and \textit{evaluate}. Research outputs, also known as \textit{artifacts}, are created with the purpose of solving a specific problem that has yet to be solved \cite[78]{hevner_design_2004}, while research activities are the processes by which those artifacts are designed to meet a criterion \cite[79--80]{hevner_design_2004}. \citeauthor{hevner_design_2004} \cite{hevner_design_2004} argued that, in fact, truth and utility are inseparable \cite[80]{hevner_design_2004}. This statement is the basis for their approach, which extends the original \gls{dsr} framework \cite{march_design_1995} presented by \citeauthor{march_design_1995} \cite{march_design_1995}, by setting the following practical guidelines: \emph{(i)} Design as an Artifact, \emph{(ii)} Problem Relevance, \emph{(iii)} Design Evaluation, \emph{(iv)} Research Contribution, \emph{(v)} Research Rigor, \emph{(vi)} Design as a Search Process and \emph{(vii)} Communication of Research. Summarizing, \gls{dsr} requires the creation of a novel, innovative, purposeful artifact, that is formally defined, for application on a specific problem domain, created through a search process to find an effective solution and, finally, communicated effectively to the community \cite[82]{hevner_design_2004}.

\citeauthor{peffers_design_2007} \cite{peffers_design_2007} present a \gls{dsrm}, extending previous research \cite{march_design_1995, hevner_design_2004}, \textit{"for the production and presentation of \gls{ds} research in \gls{is}"} \cite[3]{peffers_design_2007}. This research expanded on previous concepts by proposing a missing procedure \cite[7]{peffers_design_2007} that would connect both the principles and practical guidelines existing in \gls{ds} to create a formal research methodology. With that in mind, they have proposed a series of activities that form a process by which to conduct \gls{dsr}: \emph{(i)} Problem Identification and Motivation, \emph{(ii)} Define the Objectives for a Solution, \emph{(iii)} Design and Development, \emph{(iv)} Demonstration, \emph{(v)} Evaluation and \emph{(vi)} Communication. This set of activities aligns well with the practical rules proposed by \citeauthor{hevner_design_2004} \cite{hevner_design_2004} and, as \citeauthor{peffers_design_2007} argue, with most of the literature conducted thus far. Although the procedure is presented sequentially, the authors point out that \textit{"there is no expectation that researchers would always proceed in sequential order from activity one through activity six"} \cite[14]{peffers_design_2007}, which provides flexibility while conducting research while, at the same time, providing a rigorous process to follow. The importance of this work is that it takes the practical guidelines and theoretical principles established previously, turning them into a process for direct application on research.

We have aimed, while conducting the research presented in this thesis, to follow what has been proposed by \citeauthor{peffers_design_2007} \cite{peffers_design_2007} while, at the same time, keeping in mind the practical guidelines laid out by \citeauthor{hevner_design_2004} \cite{hevner_design_2004}. Both papers are widely cited contributions to the body of knowledge, with case studies that exemplify applicability to conducting \gls{dsr}. They have matured through years of refinement and have been wildly used by the \gls{is} research community, making them a suitable choice for a methodology to follow. To exemplify the usage breadth of these approaches, we can verify that they have been used or suggested, and cited, in \gls{is} research on supply chain on the internet of things \cite{geerts_supply_2014}, quality of business processes \cite{heidari_quality_2014}, modeling of resources and resource management \cite{speitkamp_mathematical_2010} and service systems engineering \cite{bohmann_service_2014}.

To demonstrate the application of the chosen methodology throughout this thesis, we will take the same approach as \citeauthor{peffers_design_2007} \cite{peffers_design_2007} by describing how we have fulfilled each of the designated activities. In each of these paragraphs, we aim at mentioning the respective \gls{dsr} guidelines behind the \gls{dsrm} activity, where appropriate. Although the authors have argued that this methodology of research need not be conducted sequentially, most of what has been described in this thesis has, indeed, been done in a sequential fashion. Wherever appropriate, we also relate each chapter back to the procedure described here.

\textbf{Problem Identification and Motivation.} In Chapter \ref{chap:intro}, we have introduced our thesis by motivating the existence of a specific problem while, at the same time, describing, and justifying, why a solution to this problem is needed. We have exposed a knowledge gap and the relevance of this problem. We justified the solution to the problem by using examples from the past and by connecting this problem domain with future trends, both technological and societal, showing how the problem will be aggravated by the trends in \gls{is} we have been witnessing. We have taken this perspective a step further, by dissecting in Chapter \ref{chap:related} where that knowledge gap is, in relation to the existing literature. This approach also relates correctly with what has been proposed by \citeauthor{hevner_design_2004} \cite{hevner_design_2004} by justifying the \textit{Problem Relevance} of this type of research. In Chapter \ref{chap:intro}, we also explicitly define our problem statement, to make it as clear as possible.

\textbf{Define the Objectives for a Solution.} This is, also, achieved in Chapter \ref{chap:intro}, where we discuss the objectives for this thesis. In this case, we define what a solution should accomplish in order to be relevant to our research question and our identified problem. According to what \citeauthor{peffers_design_2007} \cite{peffers_design_2007} proposed, we define qualitative objectives \cite[13]{peffers_design_2007}, in the sense that we provide a description of new artifacts and how they should solve the identified problem. By now, we are already abiding by the \textit{Design as an Artifact} guideline \cite{hevner_design_2004} due to the fact that we propose artifacts, focused on a given problem domain, to address an existing problem \cite[82]{hevner_design_2004}.

\textbf{Design and Development.} This section of the process can be found throughout Chapter \ref{chap:study} to Chapter \ref{chap:implementation}. Initially, high-level artifacts were designed, both as models and methods, of what a possible solution to the problem should accomplish. Those artifacts were then used in a conducted study, in order to assess what would be the perceptions of users to our new artifacts. Would this idea be feasible? Would it make an understandable and reasonable solution for the problem at hand? Was the identified problem recognized by users? These were the questions we were after, to validate our initial hypothesis. This study is also a property of social and informational resource \cite[6]{peffers_design_2007}, as defined by \citeauthor{peffers_design_2007} \cite{peffers_design_2007}, to further reinforce the research conducted. After that, we took the feedback into consideration, by designing lower-level artifacts, in the form of models, which we used to define methods and then an instantiation, in the form of a prototype. These are the artifacts created during our research, following this methodology, in order to evolve from the stage of objectives to creating an actual artifact that represents a solution. Our approach was mindful of: \emph{(i)} the definition that \textit{"a design research artifact can be any designed object in which a research contribution is embedded in the design"} \cite[13]{peffers_design_2007} and \emph{(ii)} of the \textit{Research Contributions} guidelines \cite[87]{hevner_design_2004}, such that all our actions have been towards generating artifacts, \textit{"new and interesting contributions"} \cite[87]{hevner_design_2004}, with a purpose that suits our problem domain.

\textbf{Demonstration \& Evaluation.} We have decided to justify both of these sections together because, in all truthfulness, a demonstration is, in itself, an aspect of evaluating a given artifact. \citeauthor{peffers_design_2007} \cite{peffers_design_2007} explain exactly that duality by saying that \textit{"solutions vary from a single act of demonstration (...) to prove that the idea works, to a more formal evaluation (...) of the developed artifact."} \cite[13]{peffers_design_2007}, when presenting both sections. The solution has been demonstrated by describing a simulation of the problem, applying the solution to it, running the \textit{instantiation} and further analyzing how the solution resolves the problem, as described in the simulation. This method - the simulation - is an approved method of evaluating as per \citeauthor{hevner_design_2004} \cite[13]{hevner_design_2004}. Furthermore, besides the experimental simulation, we applied analytical methods by evaluating the solution through a collection of guidelines \cite{hu_guidelines_2012}, specific to the problem domain of access control solutions, and descriptive methods, by way of an informed argument \cite[86]{peffers_design_2007}. Due to the fact that this research finds itself at the intersection of several different domains, and is, at the same time, in a recent research field, it becomes difficult to achieve a quantitative evaluation (e.g. performance evaluation) of each of the artifacts. We provide an exploratory qualitative analysis, based on the collection of guidelines mentioned previously. A more thorough quantitative analysis can, nonetheless, be achieved or planned, as outlined in Chapter \ref{chap:conclusion}.

\textbf{Communication.} Throughout the development of this research, communication has been one of the established priorities. The research described in this thesis has been communicated, for the most part, with researchers, and less with practitioners. \citeauthor{peffers_design_2007} \cite{peffers_design_2007} suggest that all aspects of the research should be communicated, from "the problem and its importance" \cite[14]{peffers_design_2007}, all the way to the remain aspects of the artifact, such as design, novelty, utility and evaluation. While communication was weaker when it comes to practitioners, we have focused our efforts on publishing our research, in scientific conferences and chapters in edited books. These publications focus on communicating most of what we outline and describe in this thesis, from problem identification and motivation to artifact evaluation. The chapter described access control challenges in enterprise ecosystems and possible solutions through blockchain-based technologies \cite{bryan_christiansen_access_2018}, whose main content was derived from the extensive analysis of the state-of-art presented in this thesis (Chapter \ref{chap:related}), description and motivation of the problem. Apart from this chapter, components of the research presented in this thesis have been submitted as conference research papers for \emph{(i)} \textit{\gls{acmsac19}}. The submission for \gls{acmsac19} focused on the work conducted throughout Chapter \ref{chap:study}. We would like to make another submission to share the work described throughout Chapter \ref{chap:design} and Chapter \ref{chap:implementation}.

By now, it should be clear what is the chosen methodology - \glsdesc{dsrm} \cite{peffers_design_2007} - for the research presented in this thesis. It was also described how the contents of this thesis, in each chapter, intersect with the chosen methodology, both in theory and practice.