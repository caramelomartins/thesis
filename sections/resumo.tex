\section*{Resumo}

Proteger informações confidenciais ou privadas é de extrema importância. Perdas de informação e partilha de informação confidencial podem ter impactos legais, de reputação e financeiros sérios, para indivíduos e organizações. Ao mesmo tempo, o nosso panorama tecnológico é cada vez mais complexo e distribuído, sendo cada vez mais difícil proteger o acesso à informação. Uma demonstração particular dessa situação pode ser encontrada em instituições que fornecem certificados de qualificações, como as Universidades, que, ao longo do tempo, têm vindo a aumentar esforços, para desativar produtores de certificados falsos, existentes \emph{online}. As Universidades trabalham num ambiente onde a validação de credenciais é essencial, mas esporadicamente acontece e, quando ocorre, são necessárias interacções entre várias partes. Existe, por isso, uma lacuna, entre as necessidades de ambientes tecnológicos modernos, distribuídos e complexos, no que diz respeito ao controlo do acesso à informação, e o nível ao qual as soluções clássicas de controlo de acesso podem atender a essas necessidades. Esta tese explora a utilização de \emph{blockchains} privadas, como veículos tecnológicos para a descentralização do controlo de acessos, aplicado neste caso de uso específico. Esta tese propõe \texttt{Blocked}, um sistema que permite o controlo de acesso descentralizado, através de uma \emph{blockchain} privada, para emissão, partilha e gestão de certificados educacionais. Uma avaliação deste sistema demonstra que o mesmo pode ser considerado um sistema de controlo de acessos adequado e funcional, com melhorias sobre as soluções descentralizadas existentes para o mesmo problema.

\vfill

\noindent \textbf{Palavras-Chave:} Controlo de Acessos, Segurança, Privacidade, Blockchain, Sistemas Descentralizados 