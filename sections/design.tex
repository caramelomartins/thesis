\chapter{Design and Architecture}
\label{chap:design}

We expose, in Chapter \ref{chap:intro}, that the purpose of this dissertation is to assess whether permissioned blockchains can be a solution for decentralizing access control in the context of educational certificate issuance, sharing and verification. We have taken a very small step towards that objective, by studying how users perceive blockchain, both in terms of security and complexity. To build upon that, we design and architect a system, which we implement in Chapter \ref{chap:implementation}, to prove our hypothesis. After all, it will be impossible to evaluate our initial research question if there are no artifacts, as per the \gls{dsrm}, that we can use to evaluate our proposed approach. We've envisioned this system being applied to an educational context, in which Universities, or Educational Institutions, are able to \textit{initially} deploy a network of nodes, as the blockchain \textit{genesis}, that will be maintained and run by other nodes - Students - that join and leave the network as time passes.

In this chapter, we describe the conceptual design and architecture behind our proposed system -\texttt{Blocked}. Initially, we review the \textit{stakeholders} that will interact with the system, in Section \ref{sec:design-entities}. Section \ref{sec:design-architecture} explains the system's technological architecture that guided the implementation of the system and it relates to the context in which we are working. This section establishes how the network, state and transaction architecture operate, as well as how to perform control of access to information. Finally, Section \ref{sec:design-interaction} explains the functional architecture of the system, by describing operations permitted in the system and how users are expected to interact with it.

\section{Stakeholders}
\label{sec:design-entities}

\texttt{Blocked} is designed to be used by the following stakeholders: \textit{students}, \textit{institutions} and a third entity which we establish as being a \textit{verifiers}. According to the essence of our study, presented in Chapter \ref{chap:study}, where we present the same high-level stakeholders, with a \textit{verifier} being the \textit{Recruiter}. In this case, \textit{students} are stakeholders that will be issued certificates and that control access to those certificates. \textit{Students} are also expected to be an active part of the network, maintaining one node participating in the network - e.g. being a peer. Ultimately, \textit{Students} will also share the certificates for verification by another entity - in our previous study the \emph{Recruiter}. \textit{Institutions} are the entities responsible for issuing the certificates and, if needed, dealing with revocations of certificates. They are expected to only need to interact with the \textit{students}. At the same time, \emph{institutions} are also participants in the existing network, as we describe in Section \ref{sec:design-network}. Finally, \emph{verifiers} are the entities with which certificates are shared, by the \textit{students}, and whose role in the system is that they want to verify a given certificate, shared by another stakeholder, exists and is truthful. These stakeholders are not required to participate in the validation network, since their role is merely of fetching and visualizing information, but they are required to connect to the network in order to perform verification.


\section{Technological Architecture}
\label{sec:design-architecture}

In order to understand the technological architecture of the system, we explain the inner workings of the network (Section \ref{sec:design-network}), how state is maintained throughout the network (Section \ref{sec:design-state}), how transactions are defined, processed and validated (Section \ref{sec:design-transactions}), and, finally, how management of access to information is performed (Section \ref{sec:design-ac}).

\subsection{Network}
\label{sec:design-network}

In \texttt{Blocked}, each stakeholder is expected to be a node in the network, apart from the \textit{verifier}, for a given period of time. There are two reasons for this: \emph{(i)} each node maintains its own copy of the global state of the network, which means, at least, two nodes \textbf{must} exist at any given time; and \emph{(ii)} in order to manage certificates, one must be connected to the network. At the same time, each peer is both a producer - issuing new transactions - and a validator - because it is a component of the validation network. The network is composed of a collection of nodes. The nodes interact with each other through a peer-to-peer network and maintain the global state by using a consensus algorithm - namely, \gls{poet} \cite{intel_poet}.

Given that \texttt{Blocked}'s backbone is a blockchain, and given blockchain's inherent decentralization, the appropriate way for a stakeholder to access information on the blockchain is effectively to consult a copy of the global state. There are two possible ways of doing this: either by consulting a copy through a proxy (such as REST API pointing to a copy, on another node) or possessing a local copy of the global state of the network on a stakeholder's own node. The latter is the approach designed in \texttt{Blocked}. When a stakeholder needs to access information on the network, it should connect a node to the network and download a local copy of the state.

At the same time, given that managing a certificate requires submitting transactions and batches to the blockchain, to be processed by the validators, a stakeholder that needs to modify the existing state of the network needs to be connected to the network. This justifies our second statement that \emph{"in order to manage certificates, one must be connected to the network"}.

There are two levels to our network. Each node is necessarily a peer in the network but, due to permissioning capabilities of the blockchain (detailed in Section \ref{sec:design-ac}), not necessarily a validator node in the network. One might be able to connect to the network as a node, and read information from it, and not be able to validate new blocks because it is not a participant on the validation network. The network allows for information to be shared between all nodes and, at the same time, for nodes to validate on new blocks consensually.

\subsection{State}
\label{sec:design-state}

We define \textit{state} as the information stored in a particular address of the blockchain. The global state of the system is the collection of all addresses currently maintained, with information, in the blockchain. The global state in \texttt{Blocked} is represented through a Radix Merkle tree. It can be modified by the validators and each node maintains its own copy of the global state. Validator nodes are responsible for ensuring that the global state is the same for all participants in the network. In \texttt{Blocked}, a state entry - an address - consists of the properties listed in Table \ref{tab:stateProperties}.

{
\renewcommand{\arraystretch}{1.5}%
\begin{table}[htb]
	\centering
	\caption{State Properties}
	\label{tab:stateProperties}
	\begin{tabular}{p{0.15\textwidth}|p{0.75\textwidth}}
		\hline \bf Property  & \bf Description                                                                                                                             \\ \hline
		\texttt{id}          & Represents the identifier of a given educational certificate. A version 4 \gls{uuid} generated for each certificate issuance.               \\ \hline
		\texttt{certificate} & Represents an encrypted issued certificate and should include \texttt{issuer}, \texttt{recipient}, \texttt{issued\_at} and \texttt{active}. \\ \hline
		\texttt{owners}      & List of public keys that identify both the issuer and recipient of the certificate.                                                         \\ \hline
		\texttt{permissions} & List of subjects that are allowed access to read the information.                                                                           \\
		\hline
	\end{tabular}
\end{table}
}

A state entry is stored at a particular address. An address uniquely identifies where information is stored in the global state - i.e. in the Radix Merkle tree. By fetching for a particular address, a client is able to fetch a particular piece of information stored in the global state thus being able to read the bytes, we will see in Section \ref{sec:design-ac} how reading bytes is different from accessing the information. Addressing then, in our use case, is the process of determining the address of a particular certificate - where its information is stored. Addresing is done through a process that is deterministic and must respect the following rules:

\begin{itemize}
	\item An address consists of an hex-encoded string of 70 characters.
	\item An address \textbf{must} always start with \emph{"49cb48"}. This value represents the first 6 characters of a SHA-512 hash of the string \emph{"blocked"} - the lowercase name of our system.
	\item Remaining 64 characters of an address \textbf{must} be the first 64 characters of a SHA-512 hash of the certificate property \texttt{id}.
\end{itemize}

This addressing scheme enables a client node, and validators, to always generate the correct address for a given certificate, thus being able to access blockchain's state at that address. Following these rules is the only way of accessing the correct information for any given certificate.

\subsection{Transactions and Batches}
\label{sec:design-transactions}

Transactions allow nodes to modify the global state of the system. These are submitted to the validators and processed. If valid, the state is modified. If invalid, the new blocks will be rejected. Transactions are created by the clients in each node. In \texttt{Blocked}, transactions are wrapped in batches, with each batch assumed to have one and only one transaction. Transactions in \texttt{Blocked} allow clients to issue, share and revoke educational certificates. The structure of the transaction when it is being submitted is different from what has been presented in the global state, being dependent on the operations, presented in Section \ref{sec:design-interaction}. Each transaction payload should include the operation it is trying to perform. Allowed operations are \texttt{issue}, \texttt{revoke}, \texttt{grant\_access} and \texttt{revoke\_access}. Each transaction payload should also include a \texttt{data} property, variying from operation to operation. When performing an \texttt{issue}, a client should send the \texttt{data} with the same structure as the global state structure. When performing \texttt{revoke}, a client should send an \texttt{id} and \texttt{certificate} inside \texttt{data}. When performing either \texttt{grant\_access} or \texttt{revoke\_access}, a client should send the permission object to append or the identifier of the subject whose permissions are to be removed.

\subsection{Access Control}
\label{sec:design-ac}

Controlling access to the information, in \texttt{Blocked} can be achieved in two separate layers. Due to the permissioning nature of the system, we can configure which validators are allowed to modify state in the network. At the same time, there's an access control mechanism above the network layer that is used to guarantee that, even though each node has its own copy of the information, data cannot be correctly read without the relevant permissions.

\subsubsection{Network Permissioning}

On validator nodes, we can define two separate levels of controlling modification to state: transactor-level and validator-level. At the transactor-level it is possible to control what public keys are allowed to submit batches and transactions. This type of permissioning, allows control over what identities are allowed to modify the state of the network, which can be relevant, for example, if a malicious node tries to submit new malicious information to the system. The latter level of permissioning, at the validator-level, allows the configuration of public keys as the only keys that are able to participate in the validation network, therefore controlling the subjects that are able to perform consensus and validate blocks thus modifying state. These permissioning capabilities are one of the key features of using a permissioned blockchain, instead of a permissionless blockchain.

\subsubsection{Cryptography-based Permissions}

Apart from the network-based access control, the system supports some safeguards with regards to protection of the information being stored in the network. Initially, when a certificate is submitted for issuance, the certificate information is encrypted with an AES symmetric key. A new key is generated each time a certificate is submitted and consists of 32 random bytes. This guarantees that the certificate is already submitted in an encrypted state to the blockchain, avoiding a potential \gls{mitm}, that would allow a third-party to read the information. At the same time, permissions to read the information, stored in the blockchain, are created through encrypting the symmetric key with an asymmetric public key, corresponding to the subject that shall be allowed access. A list of permissions of this type is kept in the blockchain together with the encrypted information. This method effectively avoids any access by any subject that is unable to decrypt the symmetric key and, therefore, unable to decrypt the encrypted information, while at the same time avoiding that anyone with a copy of the state reads the information.

\section{Functional Architecture}
\label{sec:design-interaction}

In order to be able to use \texttt{Blocked}, a subject needs to comply with the minimum requirements for setting up a node (Section \ref{sec:design-setup}), and be part of the network. The remaining sections describe the actions that can be performed with \texttt{Blocked}.

\subsection{Setting Up}
\label{sec:design-setup}

For a user to use \texttt{Blocked}, it needs to have an identity. Identities are defined in two ways, both with asymmetric cryptography: a DSA-based identity and an RSA-based identity. \gls{dsa} \cite{cameron_f._kerry_digital_2013} is used for signing the batches and transactions while RSA \cite{rivest_method_1978} is used for encryption of the symmetric key. While RSA \cite{rivest_method_1978} can be used for both encryption and decryption, DSA-based signatures are supported by most blockchain platforms (such as Bitcoin and Ethereum) for signing batches and transactions, which makes it easier for system implementation.

Users should generate their own keys before trying to join the network. They should generate an ECDSA \cite{johnson_elliptic_2001} key pair with secp256k1 \cite{secp256k1}, as the parameters of the elliptic curve, and store an hexadecimal representation of those keys, in order for them to be used with \texttt{Blocked}. The same should be done for the RSA keys \cite{rivest_method_1978}. The users should generate an RSA \cite{rivest_method_1978} key pair according to PKCS\#1 OAEP \cite{moriarty_pkcs_2016} and store those keys for future use in \texttt{Blocked}. After these steps are complete, users are able to identify themselves within the system thus are in a position to start using it.

In order to run a network, one of the nodes, and only one, should generate a genesis block for the blockchain, as described in the beginning of this chapter. This means that, in broad terms, one of the nodes would have to be responsible for starting the actual network. From that moment on, after other nodes connect, that node will have the same relevance as the remaining nodes.

\subsection{Issuing a Certificate}

In \texttt{Blocked}, any user with permission is able to generate a new certificate. For the sake of simplicity, let us call that user an issuer. The following steps should be executed:

\begin{enumerate}
	\item The issuer should ask for, or the recipient should share, the public keys for both the RSA-based and DSA-based identities. This will be needed to encrypt the information and identify who's receiving the certificate, and who's issuing it.
	\item The issuer should then proceed to generate and submit a transaction, wrapped in a batch, and submit it to the network. The transaction should represent the \texttt{issue} operation.
	\item The transaction is processed and one of the following outcomes should happen: \emph{(i)} the transaction is accepted and the state is updated or \emph{(ii)} the transaction is rejected and the state remains the same.
	\item After that, the issuer should either communicate the identifier of the newly generated certificate to the recipient. If the transaction was rejected, and the issuance cannot be completed, the recipient should also be advised of that.
\end{enumerate}

It is important to mention that the certificate, and consequently the state's address, can solely be identified through the identifier. If that information is lost, it will not be possible to recover the information, unless through iterating all transactions and trying to decrypt all certificates, until the correct one is found.

\subsection{Revoking a Certificate}

Revoking a certificate can be accomplished by both the issuer or recipient of a certificate. Both have control to revoke a given certificate. The following steps should be accomplished:

\begin{enumerate}
	\item A transaction should be submitted to the network with the identifier of the certificate to revoke. The transaction should represent the \texttt{revoke} operation.
	\item The network validates that the transaction has been submitted by either the issuer or receiver of the certificate and modifies the state, revoking the certificate. Otherwise it will reject the transaction.
\end{enumerate}

\subsection{Granting/Revoking Access}

One of the main purposes of \texttt{Blocked} is to enable the control of acess to the certificates stored in the network. Only the recipient of the certificate has the ability to control the access to certificates and change it. To do that, the following steps should be executed.

\subsubsection{Granting}

\begin{enumerate}
	\item The recipient should request, and the subject, to be granted access, should provide, both public keys for the DSA-based and RSA-based identity.
	\item The recipient will submit a new transaction to the network, containing the symmetric key encrypted with the RSA public key shared by the subject. The transaction should represent the \texttt{grant\_access} operation.
	\item The network validates that the transaction is valid, including validating the transaction signer as being the recipient, and modifies the state, appending the permission to existing permissions. Otherwise it will reject the transaction.
\end{enumerate}

\subsubsection{Revoking}

\begin{enumerate}
	\item The recipient will submit a new transaction to the network, with the DSA-based identity shared by the subject, when the access was granted. The transaction should represent the \texttt{revoke\_access} operation.
	\item The network validates that the transaction is valid, including validating the transaction signer as being the recipient, and modifies the state, removing the permission from existing permissions. Otherwise it will reject the transaction.
\end{enumerate}

\subsection{Viewing a Certificate}

Contrary to what has been presented thus far, viewing a certificate does not require the submission of a new transaction to the network. It is not enough to just look at the current state either because the information is encrypted. For an actor to view information pertaining to a given certificate, it should identify which address in the network stores the data for that certificate (through the certificate's identifier). It should then proceed to decrypt the existing permissions' list until it can decrypt one of those permissions and obtain the symmetric key. With the symmetric key decrypted, the actor should decrypt the certificate's information and will be able to view the certificate. This operation will only be available if access has been granted to that specific actor and is independent on the certificate being active or revoked.