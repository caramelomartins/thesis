\chapter{Conclusion}
\label{chap:conclusion}

\section{Achievements}
\textcolor{red}{TODO}

\section{Future Work}
\textcolor{red}{TODO}

This section describes the limitations found with the present study and discusses possible alternatives to mitigate this limitation in further research.

Firstly, it is important to note that this is a small sample, less than 50 respondents. This situation can give rise to erroneous results due to bias sampling. At the same time, although we have some diversity, we would like to decrease the majority of one specific field of the others and gather a more diverse set of respondents. This limitation can be overcome by reproducing the questionnaire on larger, more diverse samples. The fact that we are the subjects we are studying eschew heavily towards a more technically-driven professional area is also a reason for concern because it doesn't allow for analyzing conclusive tendencies without future studies.

Secondly, we would also like to leave a note regarding the possibility of a learning bias. Although we have shuffled our scenarios, we have only done that with 2 out of the 3 scenarios. In the questionnaires that were presented, the Blockchain scenario always appeared as the last scenario to be analyzed. This creates a situation in which the results are not sustainable without future studies. In order to be able to get more conclusive data, it would be important to perform different studies where more shuffling was involved, particularly with anything involving blockchain - which we have kept in the last position, in both versions. More shuffling can either mean \textit{(i)} more options with a randomized sequence or \textit{(ii)} repeting this questionnaire with just varying the last scenario and using it at the beginning, for example.

Finally, the platform used for the questionnaire had two limitations: \textit{(i)} a back button, which allowed respondents to rewrite their answers; \textit{(ii)} the impossibility to present the images of the interactions along with the questions - which might increase the confusion in respondents, by having to memorize the images.