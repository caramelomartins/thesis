\chapter{Introduction}
\label{chap:intro}

\section{Motivation}

Access control has been a subject of research since the early inceptions of the digital era, when time-sharing systems (\cite{weissman_security_1969}, \cite{graham_protection_1971}, \cite{bell_secure_1973}, \cite{lampson_protection_1974}, \cite{denning_lattice_1976}) were first conceptualized. Nonetheless, it is a subject with endless research opportunities due to continuous advances offered by the industry. Although it has been deeply researched, there's an increased perception of its importance, due to, among other things, recent and recurring data breaches \footnote{http://www.wired.co.uk/article/hacks-data-breaches-2017} \footnote{https://www.nytimes.com/interactive/2017/your-money/equifax-data-breach-credit.html} \footnote{https://www.theguardian.com/technology/2017/nov/29/uber-security-breach-london-sadiq-khan-users}, the rise of the Internet of Things (IoT) and cloud services. Increasingly complex and distributed technological ecosystems, with increasing numbers of users, demand different approaches on the subject of access control and its management. With information and resources increasingly scattered around the globe, in high-functioning distributed clusters of computational capacity, new challenges to the current access control methodologies are emerging. In an increasingly digital world, in which huge quantities of data are created each day, it is becoming a necessity to strengthen the mechanisms of access control, specifically, in terms of its security, decentralization, resilience, scalability and traceability.

A survey over existing decentralized access control solutions (\cite{miltchev_decentralized_2008}), for distributed file systems, has found issues with ease of use, scalability and management difficulties, specially over permissions' revocation. Existing access control solutions for cloud services is either centralized (\cite{calero_toward_2010}, \cite{ruj_dacc:_2011}, \cite{yu_achieving_2010}) or relying on complex Public Key Infrastructure and Key Distribution Centers (\cite{ruj_privacy_2012}, \cite{ruj_decentralized_2014}, \cite{bauer_distributed_2005}). A review of the state of art over access control in IoT (\cite{ouaddah_access_2017}) has suggested a modern approach to access control should be concerned with providing "many and diverse approaches" rather than a "one-size-fits-all approach". Ouaddah et al. (\cite{ouaddah_access_2017}), in the same research, suggest that current IoT access control solutions face 2 challenges: developing improved access control mechanisms, over the classical ones, and developing decentralized approaches to access control in IoT, in an effort to improve security and ensure privacy. Access control for the Internet has also been researched (\cite{park_rbac_1999}) by using \textit{smart certificates} over a centralized architecture. Other efforts in researching decentralized access control are either outdated for modern applications (\cite{satyanarayanan_integrating_1989}, \cite{karger_non-discretionary_1977}) or are purely theoretical (\cite{thomas_towards_1993}). Much of the existing research has been found to be centralized, lacking in implementations, for current systems, lacking in scalability capacities and traceability, focused on IoT or cloud services.

The emergence of blockchain-related technologies (\cite{nakamoto_bitcoin:_2008}, \cite{buterin_next-generation_2013}, \cite{wood_ethereum:_2014}), presents itself as an opportunity for a proper alternative to current access control mechanisms, to improve on the weaknesses mentioned above, in complex and private technology ecosystems: centralization, scalability and resilience, traceability and ease of use. A blockchain can be described as a decentralized ledger, whose data and integrity are maintained and validated, by a collection of participants, in a network, through consensus protocols. In practice, a blockchain can be seen as a decentralized database, without a centralized authority, which functions in an append-only mode, and whose data cannot be tampered with, due to the fact that each new record stores a reference for the previous record. Recently, Ouaddah et al. (\cite{ouaddah_access_2017}) point out something similar, that the use of decentralized access control through blockchain-based technologies is one of the research directions for the future - albeit for IoT, we can easily make a connection. Recent research suggests a growing number of blockchain-based applications (\cite{pilkington_blockchain_2015}, \cite{yermack_corporate_2017}, \cite{xu_blockchain_2016}, \cite{dorri_blockchain_2016}, \cite{fotiou_decentralized_2016}, \cite{augot_identity_2017}), including some early research over applying blockchain for access control mechanisms (\cite{maesa_blockchain_2017}, \cite{maesa_distributed_2017}), which is impractical from an enterprise and organizational perspective.

With what has been described above, our research domain emerges at the intersection between two separate domains. An established research domain, Access Control, with a long research history, and an emerging field that stems from a specific technology, Blockchain. The focus of this research thus is over the application of blockchain-based technologies for Access Control, its challenges and opportunities.

\section{Objectives}

As presented above, there's an increasing gap between the necessities of modern, and complex, distributed environments, in regards to control of access to data and resources, and the level to which classical access control solutions can fulfill those needs.

This research, presented in the context of what has been described above, aims to deepen the existing knowledge, with regards to access control models, and implementations, in complex and distributed technological environments, through an analysis of existing literature, a new approach and an implementation. We aim to provide a starting point, for research that is more focused on enterprise-level application of private decentralized access control solutions, over theoretical approaches. \textbf{Specifically, we want, with this thesis, to propose an inherently decentralized approach, focused on distributed file systems, that can showcase improved traceability, reliability, and security over existing approaches.} This approach aligns, from a business perspective with the practice of enterprise content management, at the document management level, within organizations. Our approach is based on using blockchain-based technologies (\cite{nakamoto_bitcoin:_2008}, \cite{buterin_next-generation_2013},\cite{wood_ethereum:_2014}) as the backbone of decentralized access control, leveraging blockchain's very own characteristics. A key aspect of our proposal, apart from being fully decentralized, is the focus on the use of private blockchains instead of using public ones (e.g. Bitcoin's - \cite{nakamoto_bitcoin:_2008} - or Ethereum's - \cite{buterin_next-generation_2013}, \cite{wood_ethereum:_2014}), mainly due to the fact that private blockchains are more suitable for enterprise environments.

Finally, we intend to produce further knowledge, in the research field, in the form of book chapters, paper submissions and poster submissions, to share the body of knowledge acquired over the course of the thesis and validate this research's results with the scientific community.

\section{Contributions}
\label{sec:contributions}

\section{Outline}

