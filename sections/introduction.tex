\chapter{Introduction}
\label{chap:intro}

\epigraph{\textit{The essence of the independent mind lies not in what it thinks, but in how it thinks.}}{Christopher Hitchens \cite{hitchens_letters_2009}}

\section{Motivation}

Our society is built on top of and quite dependent on centralized webs of trust. We share our personal information on a daily basis, either voluntarily by sharing it with trusted entities, such as banks and governments, or involuntarily, by making use of applications, or visiting websites, that collect our personal information. The institutions that form those webs of trust have sadly failed short of their responsibilities, on numerous occasions. Three interesting challenges emerge from this situation: \emph{(i)} are individuals, on their own, or as a collective, capable of enhancing their own privacy, through the use of technology \emph{(ii)} are there enhanced ways of sharing personal information, in a private and secure fashion, and \emph{(iii)} could breaking this centralization (\textit{e.g.}, decentralization) be a solution for improving the current \textit{status quo}?

At the same time, the amount of information humans have to process on a daily basis is enormous. It is also increasing very rapidly. This situation has created an environment in which it is becoming increasingly difficult for people to validate whether a piece of information is truthful or if it is incorrect. An example of this, exposed through media outlets, is the \text{misinformation war} ocurring, at the moment, throughout social media. It is also a topic of interest for researchers who have gone through lengths to study the problem of spread of news online \cite{vosoughi_spread_2018}. To provide another example, institutions that provide certifications of accomplishment, such as Universities, are also starting to become affected by this spread of falsehood, regarding digital efforts. There’s an increasing effort to shutdown fake certificate websites \cite{camilla_telegraph} such as RealisticDiplomas \cite{RealisticDiplomas} or DiplomaCompany \cite{DiplomaCompany}, among others, and, recently, there have surged news of falsified PhD diplomas being used.This example comes from a completely different perspective than the previous examples - \textit{fake news} - but shares some of same consequences: \emph{(i)} it affects the credibility of trusted institutions, \emph{(ii)} it impacts the outcome of certain situations due to a falsehood and \emph{(iii)} it can have serious legal and financial impacts for society.

To add complexity to this ecosystem, the information created is being offloaded, ever more rapidly, to third-party systems and entities, in the form of cloud storage, which requires users to manage access to it. Users are also getting more perceptive about the importance of being able to control what stored data entities have access to or not. One of the reasons, apart from privacy, for messaging applications to have embedded end-to-end encryption on their systems is that it disables the ability to read stored data. This means not only that sensitive data cannot be read but, perhaps most importantly, that onlys the both the sender and recipient of messages can read its contents. Similary, we have seen an increase in data breaches, both of sensitive personal data but also of sensitive documents, that have increased the awereness of the public to the importance of this problem.

We have presented three independent topics that are becoming increasingly relevant: \textit{privacy}, \textit{information verification} and \textit{access control}. These are societal issues which are gaining relevance in the public sphere but they are also, at the same time, organizational issues in way by which corporations can adapt and improve on these problems. On one hand, the impacts for society, individuals and society as a whole, are starking albeit less tangible. On the other hand, for organizations, the impacts of these problems can be financially, reputationally and legally hindering to an organization's well-being. For these reasons, research concerning these topics (privacy, information verification and access control) is of extreme importance and increasingly so.

\gls{ac} has been deeply researched but along with the context presented above, and the incresing perception of the public about its importance, challenges to the existing models and solutions for \gls{ac} are emerging rapidly. Increasingly complex and distributed technological ecosystems, with increasing numbers of users, demand different approaches to the subject of access
control and its management. Nonetheless, it is a subject with endless research opportunities due to the continuous advances offered in the industry which different approaches and solutions to \gls{ac}. Research over \gls{ac} was initially supported by government-founded institutions \cite{lampson_protection_1974, graham_protection:_1972, harrison_protection_1976} and directed towards specific systems \cite{weissman_security_1969,  organick_multics_1972, satyanarayanan_integrating_1989}, with mostly military purposes \cite{biba_integrity_1977, bell_secure_1973}. We have watched research gradually shifting from \gls{mac} \cite{bell_secure_1973, denning_lattice_1976, biba_integrity_1977, sandhu_lattice-based_1993} and \gls{dac} \cite{weissman_security_1969,  organick_multics_1972, graham_protection:_1972,lampson_protection_1974, harrison_protection_1976, sandhu_typed_1992} to more organizational perspectives, specially with the introduction of \gls{rbac} \cite{ferraiolo_role-based_1992, sandhu_role-based_1996}, with a variety of supporting models \cite{bertino_trbac:_2000, chakraborty_trustbac:_2006, joshi_generalized_2005, ray_lrbac:_2006, thomas_team-based_1997}  that suit differences in organizations and use cases \cite{barkley_role_1997, gavrila_formal_1998, park_rbac_1999, park_role-based_2003}. Different models continued to be developed for different \cite{park_towards_2002, kalam_organization_2003, sandhu_usage_2003}, with particular emphasis on the concept of \gls{abac} \cite{wang_logic-based_2004, yuan_attributed_2005, goyal_attribute-based_2006, wang_hierarchical_2010, yu_attribute_2010, hu_guide_2014}. For each new situation, \gls{ac} research has been able to provide a new solution to the emerging challenge. Recent examples of applications to modern problems include cloud computing \cite{wang_hierarchical_2010, yu_attribute_2010, wan_hasbe:_2012, ruj_dacc:_2011,calero_toward_2010}, \gls{iot} \cite{ouaddah_access_2017, dorri_blockchain_2017} and decentralization of access control \cite{sandhu_peer--peer_2005, sandhu_decentralized_1998, miltchev_decentralized_2008}.

By now, it might appear that all research problems in access control have been solved. Focusing on the descentralization of access control we can argue that is not the case. A survey over existing decentralized access control solutions \cite{miltchev_decentralized_2008} for distributed file systems has found issues with ease of use, scalability, and management difficulties, especially over permission revocation. Existing access control solutions for cloud services are either centralized \cite{calero_toward_2010, ruj_dacc:_2011, yu_achieving_2010} or rely heavily on complex Public Key Infrastructure and Key Distribution Centers \cite{ruj_privacy_2012, ruj_decentralized_2014,bauer_distributed_2005}. A review of the state of art over access control in IoT \cite{ouaddah_access_2017} has suggested a modern approach to access control should be concerned with providing ”many and diverse approaches” \cite[242]{ouaddah_access_2017} rather than a ”one-size-fits-all approach” \cite[242]{ouaddah_access_2017}. \citeauthor{ouaddah_access_2017} suggest that current IoT access control solutions face two main challenges: developing improved access control mechanisms over the classical ones, and developing decentralized approaches to access control in IoT in an effort to improve security and ensure privacy. Access control for the Internet has also been researched by using smart certificates over a centralized architecture \cite{park_smart_1999}. Other efforts in researching decentralized access control are either outdated for modern applications \cite{satyanarayanan_integrating_1989, karger_non-discretionary_1977} or are purely theoretical \cite{thomas_towards_1993}. Much of the existing research has been found to be centralized, lacking in implementations, for current systems, lacking in scalability capacities and traceability, focused on IoT or cloud services.

\begin{displayquote}
	\textit{Can permissoned blockchains be a viable solution for decentralizing access control and enhancing data ownership, in the context of educational certificate issuance, sharing and verification?}
\end{displayquote}

\section{Contributions}
\label{sec:contributions}

After describing the objectives of the presented research, it is now relevant to summarize, concretely, what are the contributions made in this thesis. As such, this thesis makes the following contributions:

\begin{itemize}
	\item a chapter in an edited book, describing access control challenges in enterprise ecosystems and possible solutions through blockchain-based technologies \cite{bryan_christiansen_access_2018}, whose main content was derived from the extensive analysis of the state-of-art presented in this thesis, description and motivation of the problem;
	\item an access control model based on ACLs and cryptography, to be used on enhancing access control on data stored inside a blockchain, as well as, a baseline framework for storing data and access control policies inside a blockchain, specifically on a permissioned blockchain built with Hyperledger Sawtooth;
	\item a proof-of-concept implementation for storing and managing Educational Certificates, and access control policies, based on the proposed models and framework, including a Transaction Processor that can be used, or extended, for Hyperledger Sawtooth, on different configurations and implementations, to process transactions, relevant to this domain;
	\item an exploratory statistical study, based on an online questionnaire, of user's perception of blockchain in terms of security and complexity, applied to an Educational Certificates use case;
	\item an evaluation on the viability of permissioned blockchains for decentralizing access control and enhancing data ownership, in the context of Educational Certificates issuance, sharing and verification.
\end{itemize}

Apart from these concrete contributions, components of the research presented in this thesis have been submitted as conference research papers for \emph{(i)} \textit{\gls{acmsac19}} and \emph{(ii)} \textcolor{red}{SECOND CONFERENCE PAPER}. The submission for \gls{acmsac19} focused on the work conducted throughout Chapter \ref{chap:study}, while the submission for \textcolor{red}{SECOND CONFERENCE PAPER} focused on the work conducted throughout Chapter \ref{chap:implementation} and Chapter \ref{chap:evaluation}. As a related achievement, due to the work conducted during this thesis, the author has been invited to peer-review research for \textit{\glsdesc{hicss52}}, on the topic of the \textit{Transformationl Impact of Blockchain}.

\section{Document Outline}

This document is structured throughout 7 chapters. After this chapter, we have Chapter \ref{chap:related} outlines some of the necessary background to this thesis, surveys the existing state-of-art and further motivates this thesis by pointing a research gap in the existing research. Chapter \ref{chap:methods} describes and motivates the methodology used throughout the research, while exposing the intersection between the chosen methodology and the chapters in this thesis. Chapter \ref{chap:study} presents a study, based on data collected through an online questionnaire with the aim of exploring the perception of users about the security and complexity of blockchains, which is a component of the solution presented in this thesis. Chapter \ref{chap:implementation} describes the implementation of the prototype developed, including the access control framework, selected technologies, design and architectural decisions. Chapter \ref{chap:evaluation} presents the evaluations of our prototype and discusses the utility of our artifcats through a series of analytical, experimental and descriptive methods. The evaluation methodology has been described previously in Chapter \ref{chap:methods}. Finally, Chapter \ref{chap:conclusion} analyzes what has been achieved in this thesis and where future research can be directed towards by discussing limitations and unexplored paths of research.
