\chapter{Introduction}
\label{chap:intro}

\epigraph{\textit{The essence of the independent mind lies not in what it thinks, but in how it thinks.}}{Christopher Hitchens \cite{hitchens_letters_2009}}

\section{Motivation}

Our society is built on top of and quite dependent on centralized webs of trust. We share our personal information on a daily basis, either voluntarily by sharing it with trusted entities, such as banks and governments, or involuntarily, by making use of applications, or visiting websites, that collect our personal information. The institutions that form those webs of trust have sadly failed short of their responsibilities, on numerous occasions. Three interesting challenges emerge from this situation: \emph{(i)} are individuals, on their own, or as a collective, capable of enhancing their own privacy, through the use of technology \emph{(ii)} are there enhanced ways of sharing personal information, in a private and secure fashion, and \emph{(iii)} could breaking this centralization (\textit{e.g.}, decentralization) be a solution for improving the current \textit{status quo}?

At the same time, the amount of information humans have to process on a daily basis is enormous. It is also increasing very rapidly. This situation has created an environment in which it is becoming increasingly difficult for people to validate whether a piece of information is truthful or if it is incorrect. An example of this, exposed through media outlets, is the \text{misinformation war} ocurring, at the moment, throughout social media. It is also a topic of interest for researchers who have gone through lengths to study the problem of spread of news online \cite{vosoughi_spread_2018}. To provide another example, institutions that provide certifications of accomplishment, such as Universities, are also starting to become affected by this spread of falsehood, regarding digital efforts. There’s an increasing effort to shutdown fake certificate websites \cite{camilla_telegraph} such as RealisticDiplomas \cite{RealisticDiplomas} or DiplomaCompany \cite{DiplomaCompany}, among others, and, recently, there have surged news of falsified PhD diplomas being used.This example comes from a completely different perspective than the previous examples - \textit{fake news} - but shares some of same consequences: \emph{(i)} it affects the credibility of trusted institutions, \emph{(ii)} it impacts the outcome of certain situations due to a falsehood and \emph{(iii)} it can have serious legal and financial impacts for society.

We have presented three independent topics that are becoming increasingly relevant: \textit{privacy}, \textit{verification} and \textit{access control}.

\section{Objectives}

As presented above, there's an increasing gap between the necessities of modern, and complex, distributed environments, in regards to control of access to data and resources, and the level to which classical access control solutions can fulfill those needs.

This research, presented in the context of what has been described above, aims to deepen the existing knowledge, with regards to access control models, and implementations, in complex and distributed technological environments, through an analysis of existing literature, a new approach and an implementation. We aim to provide a starting point, for research that is more focused on enterprise-level application of private decentralized access control solutions, over theoretical approaches. \textbf{Specifically, we want, with this thesis, to propose an inherently decentralized approach, focused on distributed file systems, that can showcase improved traceability, reliability, and security over existing approaches.} This approach aligns, from a business perspective with the practice of enterprise content management, at the document management level, within organizations. Our approach is based on using blockchain-based technologies (\cite{nakamoto_bitcoin:_2008}, \cite{buterin_next-generation_2013},\cite{wood_ethereum:_2014}) as the backbone of decentralized access control, leveraging blockchain's very own characteristics. A key aspect of our proposal, apart from being fully decentralized, is the focus on the use of private blockchains instead of using public ones (e.g. Bitcoin's - \cite{nakamoto_bitcoin:_2008} - or Ethereum's - \cite{buterin_next-generation_2013}, \cite{wood_ethereum:_2014}), mainly due to the fact that private blockchains are more suitable for enterprise environments.

Finally, we intend to produce further knowledge, in the research field, in the form of book chapters, paper submissions and poster submissions, to share the body of knowledge acquired over the course of the thesis and validate this research's results with the scientific community.

\section{Contributions}
\label{sec:contributions}

After describing the objectives of the presented research, it is now relevant to summarize, concretely, what are the contributions made in this thesis. As such, this thesis makes the following contributions:

\begin{itemize}
	\item a chapter in an edited book, describing access control challenges in enterprise ecosystems and possible solutions through blockchain-based technologies \cite{bryan_christiansen_access_2018}, whose main content was derived from the extensive analysis of the state-of-art presented in this thesis, description and motivation of the problem;
	\item an access control model based on ACLs and cryptography, to be used on enhancing access control on data stored inside a blockchain, as well as, a baseline framework for storing data and access control policies inside a blockchain, specifically on a permissioned blockchain built with Hyperledger Sawtooth;
	\item a proof-of-concept implementation for storing and managing Educational Certificates, and access control policies, based on the proposed models and framework, including a Transaction Processor that can be used, or extended, for Hyperledger Sawtooth, on different configurations and implementations, to process transactions, relevant to this domain;
	\item an exploratory statistical study, based on an online questionnaire, of user's perception of blockchain in terms of security and complexity, applied to an Educational Certificates use case;
	\item an evaluation on the viability of permissioned blockchains for decentralizing access control and enhancing data ownership, in the context of Educational Certificates issuance, sharing and verification.
\end{itemize}

Apart from these concrete contributions, components of the research presented in this thesis have been submitted as conference research papers for \emph{(i)} \textit{\gls{acmsac19}} and \emph{(ii)} \textcolor{red}{SECOND CONFERENCE PAPER}. The submission for \gls{acmsac19} focused on the work conducted throughout Chapter \ref{chap:study}, while the submission for \textcolor{red}{SECOND CONFERENCE PAPER} focused on the work conducted throughout Chapter \ref{chap:implementation} and Chapter \ref{chap:evaluation}. As a related achievement, due to the work conducted during this thesis, the author has been invited to peer-review research for \textit{\glsdesc{hicss52}}, on the topic of the \textit{Transformationl Impact of Blockchain}.

\section{Document Outline}

This document is structured throughout 7 chapters. After this chapter, we have Chapter \ref{chap:related} outlines some of the necessary background to this thesis, surveys the existing state-of-art and further motivates this thesis by pointing a research gap in the existing research. Chapter \ref{chap:methods} describes and motivates the methodology used throughout the research, while exposing the intersection between the chosen methodology and the chapters in this thesis. Chapter \ref{chap:study} presents a study, based on data collected through an online questionnaire with the aim of exploring the perception of users about the security and complexity of blockchains, which is a component of the solution presented in this thesis. Chapter \ref{chap:implementation} describes the implementation of the prototype developed, including the access control framework, selected technologies, design and architectural decisions. Chapter \ref{chap:evaluation} presents the evaluations of our prototype and discusses the utility of our artifcats through a series of analytical, experimental and descriptive methods. The evaluation methodology has been described previously in Chapter \ref{chap:methods}. Finally, Chapter \ref{chap:conclusion} analyzes what has been achieved in this thesis and where future research can be directed towards by discussing limitations and unexplored paths of research.
