\chapter{Introduction}
\label{chap:intro}

\section{Motivation}
\label{sec:intro-motivation}

Our society is built on on centralized institutions and systems, which act as authorities of governance, such as banks and governments. We share our personal information on a daily basis, either voluntarily by sharing it with trusted entities, such as banks and governments, or involuntarily, by making use of applications, or visiting websites, that collect our personal information \cite{debatin_facebook_2009, choi_embarrassing_2015, shilton_four_nodate}. These institutions that form the webs of trust have sadly failed short of their responsibilities, on numerous occasions \cite{gibbs_facebook_2014, ivashina_bank_2010, marthews_government_2017}. Three interesting challenges emerge from this situation: \emph{(i)} are individuals, on their own, or as a collective, capable of enhancing their own privacy, through the use of technology \emph{(ii)} are there enhanced ways of sharing personal information, in a private and secure fashion, and \emph{(iii)} could breaking this centralization (\textit{e.g.}, decentralization) be a solution for improving the current \textit{status quo}?

At the same time, the amount of information humans have to process on a daily basis is enormous \cite{hilbert_worlds_2011, lee_information_2016}. There is also a perception that the amount stored is increasing very rapidly. This situation has created an environment in which it is becoming increasingly difficult for people to validate whether a piece of information is truthful or a falsehood. An example of this, exposed through media outlets, is the \text{"misinformation war"} going on, at the moment, throughout social media platforms. It is also a topic of interest for researchers who have gone through lengths to study the problem of how news spread online \cite{vosoughi_spread_2018}. To provide another example, institutions that provide certifications of accomplishment, such as Universities, are also starting to become affected by this epidemic spreading of falsehoods. There’s an increasing effort to shutdown fake certificate websites \cite{camilla_telegraph} such as RealisticDiplomas \cite{RealisticDiplomas} or DiplomaCompany \cite{DiplomaCompany}, among others, going as far as faking PhD credentials \cite{doctoroff_tang_2010}.This example comes from a completely different perspective than the previous examples - alleged \textit{fake news} - but shares some of same consequences: \emph{(i)} it affects the credibility of trusted institutions, \emph{(ii)} it impacts the outcome of particular situations due to a falsehood and \emph{(iii)} it can have serious legal and financial impacts for society.

To add insult to injury, as well as complexity to this ecosystem, the information created is being offloaded, ever more rapidly, to third-party systems and entities, in the form of cloud storage, which requires users to manage access to it more thoroughly. Users are also getting more perceptive about the importance of being able to control access. One of the reasons, apart from privacy, for messaging applications to have embedded end-to-end encryption on their systems, is that is prevents third-parties from being able to read the information. It is a feature while at the same time being a security enhancement for the providers of the service. This means that sensitive data cannot be read but, perhaps most importantly, that only the sender and recipient of a message can read its contents. Similary, we continuous data breaches \cite{edwards_hype_2016}, both of sensitive personal data but also of sensitive documents, that have increased the awereness of the public to the importance of this problem.

We have presented three independent topics that are becoming increasingly relevant: \textit{privacy}, \textit{information verification} and \textit{access control}. These are societal issues with growing relevance in the public sphere but they are also, at the same time, organizational issues, in the way by which corporations can adapt and improve, in order to overcome these problems. On one hand, the impacts for society, individuals and society as a whole, are starking albeit less tangible. On the other hand, for organizations, the impacts of these problems can be financially, reputationally and legally hindering to an organization's well-being. For these reasons, research concerning these topics (privacy, information verification and access control) is of extreme importance and increasingly so.

\gls{ac} has been deeply researched but challenges to the existing models and solutions for \gls{ac} are emerging rapidly - some of those described above. Increasingly complex and distributed technological ecosystems, with increasing numbers of users, demand different approaches to access control and its management. Nonetheless, it is a subject with endless research opportunities due to the continuous advances offered by the industry. Research over \gls{ac} was initially supported by government-founded institutions \cite{lampson_protection_1974, graham_protection:_1972, harrison_protection_1976} and directed towards specific systems \cite{weissman_security_1969,  organick_multics_1972, satyanarayanan_integrating_1989}, with mostly military purposes \cite{biba_integrity_1977, bell_secure_1973}. We have watched research gradually shifting from \gls{mac} \cite{bell_secure_1973, denning_lattice_1976, biba_integrity_1977, sandhu_lattice-based_1993} and \gls{dac} \cite{weissman_security_1969,  organick_multics_1972, graham_protection:_1972,lampson_protection_1974, harrison_protection_1976, sandhu_typed_1992} approaches to more organizational perspectives, specially with the introduction of \gls{rbac} \cite{ferraiolo_role-based_1992, sandhu_role-based_1996}, with a variety of supporting models \cite{bertino_trbac:_2000, chakraborty_trustbac:_2006, joshi_generalized_2005, ray_lrbac:_2006, thomas_team-based_1997}  that suit differences in organizations and use cases \cite{barkley_role_1997, gavrila_formal_1998, park_rbac_1999, park_role-based_2003}. Different models continued to be developed with different aims \cite{park_towards_2002, kalam_organization_2003, sandhu_usage_2003}, with particular emphasis on the concept of \gls{abac} \cite{wang_logic-based_2004, yuan_attributed_2005, goyal_attribute-based_2006, wang_hierarchical_2010, yu_attribute_2010, hu_guide_2014}. For each new situation, \gls{ac} research has been able to provide a new solution to the emerging challenge. Recent examples of applications to modern problems include cloud computing \cite{wang_hierarchical_2010, yu_attribute_2010, wan_hasbe:_2012, ruj_dacc:_2011,calero_toward_2010}, \gls{iot} \cite{ouaddah_access_2017, dorri_blockchain_2017} and decentralization of access control \cite{sandhu_peer--peer_2005, sandhu_decentralized_1998, miltchev_decentralized_2008}. Some of these works have been more successful, and are more mature, than others.

By now, it might appear that all research problems in access control have been solved. Focusing on the descentralization of access control we can argue that is not the case. A survey over existing decentralized access control solutions \cite{miltchev_decentralized_2008}, for distributed file, systems has found issues with ease of use, scalability, and management difficulties, especially over permission revocation. Existing access control solutions for cloud services are either centralized \cite{calero_toward_2010, ruj_dacc:_2011, yu_achieving_2010} or rely heavily on complex Public Key Infrastructure and Key Distribution Centers \cite{ruj_privacy_2012, ruj_decentralized_2014,bauer_distributed_2005}. A review of the state of art over access control in IoT \cite{ouaddah_access_2017} has suggested a modern approach to access control should be concerned with providing ”many and diverse approaches” \cite[242]{ouaddah_access_2017} rather than a ”one-size-fits-all approach” \cite[242]{ouaddah_access_2017}. \citeauthor{ouaddah_access_2017} \cite{ouaddah_access_2017} suggest that current IoT access control solutions face two main challenges: developing improved access control mechanisms over the classical ones, and developing decentralized approaches to access control in IoT, in an effort to improve security and ensure privacy. Access control for the Internet has also been researched by using smart certificates over a centralized architecture \cite{park_smart_1999}. Other efforts, in researching decentralized access control, are either outdated for modern applications \cite{satyanarayanan_integrating_1989, karger_non-discretionary_1977} or are purely theoretical \cite{thomas_towards_1993}. Much of the existing research has been found to be centralized, lacking in implementations, for current systems and use cases, lacking in scalability capacities and traceability, or purely focused on IoT or cloud services. Recent literature has proposed alternatives for descentralized access control through the usage of blockchain-based solutions but this is still an understudied problems, as those solutions are either too specific or too theoretical, without having matching implementations.

Going back to one of the problems briefly described previously,institutions providing education to students, apprentices or trainees, by completion of a specific amount of learning hours, credits or practical assignments, or all of those combined, provide an achievement certificate. A certificate validates that someone has reached a level of understanding, mastery or capability that is expected by the end of the course. At the same time, these certificates are also proof, from a learner's perspective, that these activities have been successfully completed, either to share with a recruiter, as requirements for further education or simply as a certification of completion. Given this, guaranteeing validity and integrity of these certificates is important. These situations highlight that preventing certificate forgery is relevant, not purely as a theoretical problem but as a practical, recurring issue that hasn't been solved. At the same time, apart from preventing the forgery of educational certificates, there's also an overlooked importance on being able to guarantee the integrity of a given issued certificate, which is not possible currently - e.g. guaranteeing that the information of the certificate has not been modified. When we start considering in the fact that MOOC-based courses have been improving in contents, credibility and acceptance, this problem starts to emerge as having the potential to be explored hastily, with few tools to prevent this. With the rise of MOOCs and online education, this research problem of issuing, storing and sharing educational certificates, in a digital format, while maintaining an ease of use, improving security and privacy, will only become increasingly relevant. Furthermore, issuing, storing and sharing educational certificates emerges as an interesting and relevant challenge to solve for the modern digital age: \emph{(i)} due to the amount of business actors involved, such as \textit{Students}, \textit{Educational Institutions} and \textit{Recruiters}, or any other entity requiring validation of a given certificate; \emph{(ii)} due to the security considerations that need to be understood and asserted; \emph{(iii)} due to privacy concerns; \emph{(iv)} due to the increasing need of educational certificates, from the rise of MOOC-based learning; \emph{(v)} and due to the fact that it has yet to be solved. There has been some recent research aiming at solving this problem (\cite{MITCertificates}, \cite{Blockcerts}), with aspects to improve.

This problem intersects with the three aspects we have mentioned in the beginning of this section. It intersects with privacy, as we have described, because a \textit{Student}, or any given recipient of a credential might not want that information, or its personal information, to be described publicly. Currently, there are few solutions for that situation. It intersects with the concept of \textit{information verification}, as we have explored before, because verifying that a certificate is legitimate is getting increasingly relevant, and increasingly difficult at the same time. It intersects controlling access to information because, as we have seen, most solutions existing currently are either centralized or focus the responsibility of managing access control on the institutions rather than the individuals.

\section{Problem Statement}

The objective of this thesis is to perform research in order to be able to answer the following question: \textit{Can permissoned blockchains be a solution for decentralizing access control, in the context of educational certificate issuance, sharing and verification?} In order to achieve that, in alignment with what has been described in the previous section, it is important to analyze technological aspects of the problem and the human aspects.

\section{Contributions}
\label{sec:contributions}

In this thesis, we perform an exploration on both aspects. We describe a statistical study and analysis of a specific edge case of the human aspects, provide a theoretical base in which future research can be implemented on, as well as a proof-of-concept prototype that applies a blockchain-based solution to this problem.

With the approach taken in this thesis, we have introduced knowledge regarding the human aspects of blockchain technology to the existing literature, have created a theoretical base that supports a solution for our research question and instantiated the theoretical base, through a prototype, showing that it can in fact be implemented. We have also provided an argument for the applicability and relevance of these results in other societal and organizational problems.

Concretely, this thesis makes the following contributions: an exploratory statistical study of user's perception of blockchain, in terms of security and complexity; the design and architecture of a system for descentralization of access control, with a permissioned blockchain, for our use case; a proof-of-concept implementation for storing, sharing and managing Educational Certificates; and an evaluation on permissioned blockchains as a solution for decentralizing access control, in the context of Educational Certificates issuance, sharing and verification.

As a related achievement, due to the work conducted during this thesis, the author has been invited to peer-review research for \textit{\glsdesc{hicss52}}, on the topic of the \textit{Transformationl Impact of Blockchain}.

\section{Document Outline}

This document is structured throughout 7 chapters. After this chapter, we have Chapter \ref{chap:related}. That chapter outlines some of the necessary background to this thesis, surveys the existing research and further motivates this thesis by pointing a research gap in the existing research. Chapter \ref{chap:methods} describes and motivates the methodology of the research, while exposing the intersection between the chosen methodology and the chapters in this thesis. Chapter \ref{chap:study} presents a study, based on data collected through an online questionnaire with the aim of exploring the perception of users about the security and complexity of blockchains, a component used in the design of the solution presented in this thesis. Chapter \ref{chap:design} discusses the conceptual design and architecture that underly the implementation of the solution that has been developed. Chapter \ref{chap:implementation} describes the implementation of the prototype developed, including the access control framework, selected technologies, design and architectural decisions. Chapter \ref{chap:evaluation} presents the evaluations of our prototype and discusses the utility of our artifcats through a series of analytical, experimental and descriptive methods. Finally, Chapter \ref{chap:conclusion} analyzes what has been achieved in this thesis and where future research should focus by discussing limitations and unexplored paths of research.
