\chapter{Evaluation}
\label{chap:evaluation}

\section{Proof-of-Concept Benchmark}
\label{sec:eval-benchmark}

\section{Proof-of-Concept Analysis}

\section{Human Perception of Blockchain}

\section{Access Control Evaluation}

In this section, we evaluate our system through the metrics proposed in \cite{hu_guidelines_2012} which is a continuation on what has been presented in \cite{hu_assessment_2006}. Leveraging these metrics allows us to have an understanding of what the access control system is capable of executing, against some known organizational metrics. As \citeauthor{hu_guidelines_2012} \cite{hu_guidelines_2012} establish, \emph{"the quality metric should be evaluated based on the specific needs for the AC policy"} \cite[25]{hu_guidelines_2012}. In our context, what this means is that rather than evaluating every single metric, with several of them being clearly inadequate, we should evaluate the subset of policies that seems adequate to our use case. We have provided a summary of all the metrics we are evaluating in Table \ref{tab:qualityMetrics}. Questions in 4.1.7 and 4.1.9 have been rephrased to aggregate the existing section's questions into one question.

{   
    \renewcommand{\arraystretch}{1.5}%
    \begin{table}[htb]
        \centering
        \small
        
        \caption{Quality Metrics for Evaluation}
        \label{tab:qualityMetrics}
        \begin{tabular}{l|l}
            \hline \bf Section & \bf Metric Items \\ \hline
            \multicolumn{2}{l}{Administration Properties} \\ \hline
            4.1.1            & \llap{\textbullet} Does the AC system log denied access requests?  \\
                & \llap{\textbullet} Does the AC system log granted access requests? \\ \hline
            4.1.2 & \llap{\textbullet} Does the system provide query/display for privileges discovery? \\
                & \llap{\textbullet} Does the system provide graphic display? \\ \hline
            4.1.3 & \llap{\textbullet} How many steps are required for assigning a privilege? \\
                & \llap{\textbullet} How many steps are required for removing a privilege? \\ \hline
            4.1.4 & \llap{\textbullet} Is the AC system capable of logical expression for rule specification? \\ \hline
            4.1.5 & \llap{\textbullet} Does the AC system allow policy expiration assignment? \\ 
                & \llap{\textbullet} Does the AC system provide policy deployment or activation verification? \\
                & \llap{\textbullet} Does the AC system allow runtime policy rule change? \\ \hline
            4.1.7 & \llap{\textbullet} How is the AC enforced? \\ \hline
            4.1.8 & \llap{\textbullet} Does the AC system support multiple hosts via network? \\ \hline
            4.1.9 & \llap{\textbullet} What is the scope of data control? \\
            \hline \multicolumn{2}{l}{Enforcement Properties} \\ \hline
            4.2.2 & \llap{\textbullet} Is the AC system capable of bypassing policy rules for critical AC decisions? \\ \hline
            4.2.3 & \llap{\textbullet} Is the AC system capable of enforcing the least privilege principle? \\ \hline
            4.2.4 & \llap{\textbullet} Is the AC system capable of specifying Static SoD rules? \\ \hline
            4.2.5 & \llap{\textbullet} Does the AC system provide safety check capabilities to prevent leaking of permissions? \\ \hline
            4.2.6 & \llap{\textbullet} Does the AC system allow configuring the granularity of controlled objects? \\ \hline
            4.2.7 & \llap{\textbullet} Does the AC system support existing AC standards? \\ 
            \hline \multicolumn{2}{l}{Performance Properties} \\ \hline
            4.3.1 & \llap{\textbullet} Does the response time of granting an access request meet the organization’s requirement? \\ \hline
            4.3.2 & \llap{\textbullet} Does the AC policy retrieval and deposit meet the organization’s requirements? \\ \hline
            4.3.3 & \llap{\textbullet} Does the AC system provide an AC policy distribution mechanism? \\ \hline
            4.3.4 & \llap{\textbullet} Can the AC system be integrated with or support identification authentication systems? \\
            \hline \multicolumn{2}{l}{Support Properties} \\ \hline
            4.4.2 & \llap{\textbullet} Is the AC system capable of supporting a different OS beside the one used by the intended host(s)? \\ \hline
            4.4.4 & \llap{\textbullet} Does the AC system provide a GUI or an API for AC policy management and authoring? \\
            \hline
        \end{tabular}
    \end{table}
}

\textbf{Administration Properties.} \emph{Auditing}: \texttt{Blocked} audits the granting of access through the processing that happens inside the Transaction Processor. If the transactions submitted are approved, these are then stored in the blockchain. If, in turn, the transactions are rejected, the nodes will have a log that a given transaction has been rejected, along with the reason. At the same time, each client performs the needed validations for accessing a certificate, which means that denied read accesses won't be logged anywhere, except on the client node. The system does not provide additional log functions that can be customized and logging system failure can, also, only be done at the network level given the descentralized nature of the system. The auditing that gets stored is only per transaction, apart from all the information that can be stored in a transaction (such as allowing or revoking a given access), there are no more logs available. \emph{Capabilities Discovery}: The system has lacking capabilities discovery functionality. It is not possible to discover capabilities of a given subject due to the fact that those capabilities are stored inside each certificate, spread throughout the blockchain. It is possible to discover which subjects have access to each cetificate, nonetheless. It is not possible to discovery any more information about each capability. \emph{Ease of Privilege Assignments}: Assigning or removing a privilege is possible by executing the respective \emph{blocked} client. In both cases, it takes only one step to execute, apart from knowing the public identity of whoemever the privilege concerns. Updating a privilege is impossible. The system does not support the usage of group relations or inheritance.\emph{Specifying AC Rules}: The system does not support the specification of AC rules nor can't it handle any rule specification logic. \emph{Policy Management}: The system does not allow meta-policy information, expiration assignments, target event assignments, combinations for policies, policy distribution approval, management of policy authoritative sources or impact analysis. It does allow for target assignments, given that the default policy is directly assigned to a given target and it allows for runtime policy changes, in the sense that it allows a permission to be revoked during runtime. \emph{Delegation of Administrative Capabilities}: No policy administration delegation is allowed. The system does not support it. \emph{Flexibilities of Configuration into Existing Systems}: The access control is enforced via a combination of application logic, consensus protocols and cryptography. For viewing a certificate, a subject will need to possess the appropriate RSA private key, otherwise it won't be able to decrypt the data. For granting a permission, a subject also needs to have appropriate keys but, at the same time, needs to be either the issuer or recipient of a given certificate, otherwise the transaction will be rejected by the Transaction Processor running in the nodes. \emph{Horizontal Scope of Control}: The system supports an array of hosts, each running its own Transaction Processor, that will connect via the network. Each node will have a copy of the entire blockchain. \emph{Vertical Scope of Control}: The system covers application data that is store inside a blockchain. It does not cover any other scope of data.

\textbf{Enforcement Properties.} \emph{Policy Combination, Composition, and Constraint}: The system allows only one policy and does not allow combination or composition of policies. \emph{Bypass}: It is currently possible to bypass the system by changing the code of the Transaction Processor and running a changed version on the network. Nonetheless, it would still have to bypass the consensus of the network because it would be issuing invalid policies which would result in rejected transactions. On the viewer, it would be extremely had to bypass the system due to the fact that all the data is encrypted and secret keys, only known to the subjects, would be needed to bypass that data. Nonetheless, having those keys, one could bypass the system. \emph{Least Privilege Principle}: Every subject is considered as having no access, unless it can decrypt one of the permissions, which means they have been granted access to that certificate. Granting access to a certificate grants access to only that certificate and no other on the system. At the same time, decrypting that certificate's symmetric key will prove useless when decrypting any other certificate because every certificate is encrypted with a different, randomly generated key. \emph{Separation of Duty}: Since the system is based on a permissions-per-certificate model, it doesn't support separation of duties. \emph{Safety}: There's a universal constraint on the system that states that only subjects that have the corresponding private keys, to the public keys that were granted access to. This prevents leakage of permissions. In case a permission is compromised, that permission should be revoked during runtime. \emph{Conflict Resolution or Prevention}: The system does not support conflict resolution or prevention due to the fact that it doesn't allow policy definiton in such a way that it would generate conflicts. \emph{Operational/Situational Awareness}: The system does not support this. At the same time, this functionality doesn't fit the model that has been defined. \emph{Granularity of Control}: The system only supports defining the granularity to the certificate object, it doesn't support any other finer-grained controls. \emph{Expression Properties}: The system does not support expressing policy, rule specification languages, rule composing using standards or rule combinations. \emph{Adaptable to Evolution of AC Policies}: The system does not support adapting to evolution because it only supports a single policy.

\textbf{Performance Properties.} \emph{Response Time}: \textcolor{red}{EXPERIMENTAL EVALUATION}. \emph{Policy Repository and Retrieval}: \textcolor{red}{EXPERIMENTAL EVALUATION}. \emph{Policy Distribution}: Policy is distributed through a consensus protocol to all the nodes in the network, after each has validated the transaction, the data is updated accordingly in their blockchain. \emph{Authentication}: Authentication is performed on the basis of public key criptography, through the usage of public keys as identities.

\textbf{Support Properties.} \emph{Policy Import and Export}: The system does not support import, export or convertion of access control policies. \emph{OS Compatibility}: The system supports only Ubuntu 18.04 LTS. \emph{Policy Source Management}: The system does not use authorative sources, given that one of the goals is descentralizing the system thus having no single authority, and, for that, does not support source management. \emph{User interfaces and API}: The system provides only command line user interfaces for interaction with the system. \emph{Verification and Compliance Function}: The system does not provide verification and compliance functionalities.

\section{Comparative Analysis}

\section{Research Limitations}

It is important to note that this is a small sample, less than 50 respondents. This situation can give rise to erroneous results due to a biased sampling. At the same time, although we have some diversity, we would like to decrease the volume of one specific field when compared to others and gather a more diverse collection of respondents. This limitation can be overcome by reproducing the questionnaire on larger, more diverse samples. The fact that the subjects we are studying eschew heavily towards a more technically-driven professional area is also a reason for concern because it doesn't allow for analyzing conclusive tendencies without future studies. We aimed at reaching a bigger, more diverse sample, by speaking with subjects from different backgrounds, and publicizing the questionnaire in non-technical venues, alas we weren't as successful. At the same time, this is an exploratory study on the subject and we see this sample as being strong enough to validate potential for further studies in this area.

We would also like to leave a note regarding the possibility of a learning bias. Although we have shuffled our scenarios, we have only done that with 2 out of the 3 scenarios. In the questionnaires that were presented, the Blockchain scenario always appeared as the last scenario to be analyzed. This creates a situation in which the results are not sustainable without future studies. In order to be able to get more conclusive data, it would be important to perform different studies where more shuffling was involved, particularly with anything involving blockchain - which we have kept in the last position, in both versions. More shuffling can either mean: \textit{(i)} more options with a randomized sequence; \textit{(ii)} repeating this questionnaire with just varying the last scenario and using it at the beginning, for example. It is worth noting, nonetheless, that this situation was an attempt at an equilibrium. Given that the study consisted of 3 scenarios, we had 6 possible permutations to decide on a sequence. With 2 versions only we were maximizing the distribution by allowing for a greater number of respondents to answer each version. Had we used 6 different versions, with the same number of respondents, we would have had less 8 respondents for each version, which is definitively low. This way we were able to gather a more significant number of answers for each version.

Finally, the platform used for the questionnaire had two limitations: \textit{(i)} a back button, which allowed respondents to rewrite their answers; \textit{(ii)} the impossibility to present the images of the interactions along with the questions - which might increase the confusion in respondents, by having to memorize the images. This isn't necessarily an issue with the study itself but it is rather a technological limitation we faced that we would rather correct in future studies.