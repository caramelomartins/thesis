\chapter{Related Work}
\label{chap:related}

In this chapter we present some relevant background and work that is related with this thesis. Due to the broad nature of our research problem, as described in the previous chapter, there's a need to review several distinctive areas of interest. We start by reviewing, in Section \ref{sec:related-ac}, existing literature on \gls{ac} models and applications, discussing how they related to this research and highlighting the chosen models for our use case. Section \ref{sec:related-blockchain} reviews different blockchain approaches, dissecting their components, and describes existing literature on blockchain applications. Finally, Section \ref{sec:related-ec} reviews some of the work done on the topic of Educational Certificates - the use case of focus in this thesis. By the end of this chapter, it should be possible to realize how these concepts, and existing literature, relate to the core contributions of thesis. It should also be possible to contextualize our core research within the existing literature and, at the same time, lay out the necessary concepts to understand the forthcoming chapters.

\section{Access Control}
\label{sec:related-ac}

\glsdesc{ac} has been a core component in every technology evolution and continues to be as relevant today. It has always played a central role in technology because information is a valuable asset that must be protected from prying eyes. Due to that fact, both governments and corporations have spent considerable resources on developing \gls{ac} models and instantiations, while academia produced a broad spectrum of literature on the topic. Said literature has been produced with all sorts of approaches, from the purely theoretical to implementation-based approaches. Nonetheless, no matter what has been produced and developed previously, there are always new challenges to be faced and, with every new evolution, a new challenge emerges, which has led the field to be one of extensive study ever since the beginnings of the computer age. As more sensitive data is shared and stored by third-parties, the relevance of \gls{ac} is only increasing, rather than decreasing, creating an optimal context for researchers to be able to developed interesting new research that solves important problems.

\subsection{\glsdesc{mac}}
\label{sec:models-mac}

\glsdesc{mac} is a keystone \gls{ac} approach characterized by embedding access control policies inside systems, a central authority, rather than in external sources, for example a single subject, transforming the policy into an unchangeable entity, from the perspective of any individual \cite[23]{biba_integrity_1977}.  This indicates that policy decisions are independent from the object's owner and its owner has no control over what access rights an object has \cite[7]{hu_assessment_2006}. This approach emerged in the early stages of \gls{ac} research, in the 1970s, when research on this topic was very focused for military applications \cite{sandhu_lattice-based_1993}. A classic example is, therefore, the military setup. A user (subject) can only access documents (objects) that have the same clearance level, or lower, as the user's clearance level. \cite{hu_assessment_2006} describes this situation:

\begin{displayquote}
    (...) for example, a user who is running a process at the \textit{Secret} classification should not be allowed to read a file with a label of \textit{Top Secret}. \cite[7]{hu_assessment_2006}
\end{displayquote}

\citeauthor{sandhu_lattice-based_1993} \cite{sandhu_lattice-based_1993} provides an overview of the existing fundamental models following this approach \cite{sandhu_lattice-based_1993}: the Bell-LaPadula model \cite{bell_secure_1973} and the Biba model \cite{biba_integrity_1977}, as well as \citeauthor{denning_lattice_1976}'s axioms for information flow policies \cite{denning_lattice_1976}. These access models are commonly known as the collection of Lattice-based Access Control models, due to the fact that the sets of \textit{security classes} for the objects and subjects form what can be described as a lattice. \citeauthor{denning_lattice_1976}'s axioms coined the concept of \textit{security classes}, which can be seen as the labels described above (\emph{Secret} and \emph{Top Secret}), that define the rules that guide the logic behind the previous example. Essentially, that a subject may only access objects that are of a lower or equal \textit{security class} than the subject's \emph{security class} \cite[1-4]{sandhu_lattice-based_1993}. These axioms aren't an access control model but rather a set of guidelines that enforce security of information. The Bell-LaPadula model is essentially an access matrix (see Section \ref{sec:models-dac}), which subjects are able to modify, enforced with unmodifiable mandatory policies. This specifies a more flexible use of mandatory access control while maintaining a strict core of mandatory policies \cite[9-11]{sandhu_lattice-based_1993}. Biba's model functions in a similar way to the Bell-LaPadula model albeit with a focus on integrity levels of the objects, rather than focusing on the level of confidentiality that they should posess \cite[11-13]{sandhu_lattice-based_1993}.

\subsection{\glsdesc{dac}}
\label{sec:models-dac}

A more flexible approach, albeit potentially less secure, in a strict situation, is \glsdesc{dac}. \citeauthor{sandhu_lattice-based_1993} refers to \gls{dac} as \emph{"inadequate to enforce information flow"} \cite[8]{sandhu_lattice-based_1993}, due to its lack of constraints on copying information. \gls{dac} is characterized by allowing policies to be defined dynamically, by object owners, contrary to the inherently static policies provided with \gls{mac} \cite[23]{biba_integrity_1977}. With \gls{dac}, policy enforcement is achieved by comparing a subject and its associated groups with the resources’ permissions and group permissions. In other words, validation is performed to answer the following question: is the subject, or a group the subject belongs to, allowed access to a specific resource? As an example, this is an activity performed frequently by users in organizations when they share specific files, with specific people, with a particular collection of permissions - e.g. read but not write. In this case, the owner of the document, or an authorized subject, is dynamically modifying the existing access control policy.

Most \gls{dac} models are intrinsically related with the concept of an \emph{access matrix} \cite{graham_protection:_1972, lampson_protection_1974}. Access matrices consist of a matrix composed by objects and subjects that define a system’s access control policy – subjects can
be replaced by access domains or groups, in this instance. During policy
enforcement, access to an object is allowed if the intersection between a subject and an object bears a policy
that allows access to said object, at the level it is being requested. Along with the models proposed by \citeauthor{graham_protection:_1972} \cite{graham_protection:_1972} and \citeauthor{lampson_protection_1974} \cite{lampson_protection_1974}, there's another fundamental model that implements this concept, known as the \gls{hru} model \cite{harrison_protection_1976}. More recent models have been proposed to overcome a set of undecidable cases when performing policy enforcement, with \gls{hru}, which left it vulnerable as an access control model \cite{sandhu_schematic_1988, sandhu_typed_1992}.

Apart from access matrices, \gls{dac} offers other mechanisms for access control such as \gls{acl} and Capabilities. \gls{acl}, as the name suggests, can be thought out as a list of subject permissions, and their types, associated to an object. This model corresponds to a single row of an access matrix with all the information for each subject inside that column \cite{sandhu1994access}. The Capabilities mechanism can be understood as being an inverse mechanism to the access matrix and \gls{acl}, where permissions are stored from the perspective of the subjects and which access they have to what objects \cite{sandhu1994access}.

\subsection{\glsdesc{rbac}}

\glsdesc{rbac} emerges as a commercially focused approach to access control, contrary to \gls{dac} and \gls{mac}, which were predominantly focused on government and military applications \cite{ferraiolo_role-based_1992}. RBAC introduced the concept of roles as an intermediary abstraction between the object and subjects, in access control. With RBAC, policy definition was composed of three elements: subjects, objects, and roles \cite{ferraiolo_role-based_1992}. In this sense, RBAC is already different from the solutions described above, whose focus was solely subjects and objects. With RBAC, roles are defined as a function of, for example, employee’s job titles. Each subject is assigned a role and each object is allowed access by a set of roles. During policy enforcement, the validation performed is if the subject is a member of the defined roles to which the object allows access. RBAC made policy management easier and it was better suited for application systems that were commercially-focused instead of the military-based applications for which access control had been developed thus far \cite{ferraiolo_role-based_1995}. The reason for this suitability is due to the manner in which organizations are usually structured. Organizations, unlike the military, have employees assigned specific business roles and functions. Those roles carry with them a set of tasks and permissions that every employee, assigned to that role, must possess. Organizations can easily have dozens, hundreds or thousands of employees which makes managing policy at an individual level, painstakingly difficult through matrices or access classes. By abstracting individual employees into sets of roles, the quantity of policy definitions that are needed decreases dramatically.

Apart from the fundamental model \cite{ferraiolo_role-based_1992}, proposed by \citeauthor{ferraiolo_role-based_1992}, in \citeyear{ferraiolo_role-based_1992}, other models soon emerged such that a family of \gls{rbac} reference models was specified, known as RBAC96 \cite{sandhu_role-based_1996}. Aside from providing a formal definition of 4 models, with the base model being the one defined previously \cite{ferraiolo_role-based_1992}, this work also added the ability for roles to have hierarchy and constraints attached to them, independently or at the same time. A wide array of variations to \gls{rbac} exist with temporal constraints \cite{bertino_supporting_1996, bertino_temporal_1996} and periodic constraints \cite{bertino_access_1998}. Other variations have evolved into independent concepts that rely on the \gls{rbac} foundations such as \gls{trbac} \cite{bertino_trbac:_2000}, \gls{erbac} \cite{bonatti_erbac:_2013}, \gls{grbac} \cite{covington_generalized_2000}, \gls{tmac} \cite{thomas_team-based_1997}, \gls{lrbac} \cite{ray_lrbac:_2006} and \gls{geo-rbac} \cite{damiani_georbac:_2007}. A lot more variants of the \gls{rbac} model exist with these being but a few examples. Apart from variations on the model, research has been performed on proving that \gls{rbac} constructs are sufficiently expressive to replace mechanisms based on \gls{dac} and \gls{mac} \cite{jin_unified_2012}. This research is important because it shows how to perform modifications on legacy systems, using \gls{dac} or \gls{mac}, to more modern, potentially more suitable, access control models.

\subsection{\glsdesc{abac}}

There's one challenge the proliferation of \gls{rbac} variations uncovers which is that the world is getting increasingly more diverse in the necessities of access control, with small variations to access control being dependent either on the context of the problem domain, or the context of the system in which it is being implemented. This challenge has motivated the development of a different approach: \glsdesc{abac}. In \gls{abac}, different from what has been described above in the other approaches, access control policies can be defined with a different component, apart from subjects, objects and roles – attributes. \gls{abac} utilizes subject and object attributes (or environment attributes) in policy definition as well as in policy enforcement \cite{hu_guide_2014}. The usage of attributes increases the expressiveness of policies and allows policy enforcement to be more dependent on dynamic data - increasing the effectiveness of access control.

Given the flexible and case-to-case nature of \gls{abac}, formally defining a generalized model is not trivial, such as the ones presented for \gls{dac}, \gls{mac} and \gls{rbac}. \citeauthor{hu_guide_2014} \cite{hu_guide_2014} have attempted a generic formal definition from previous definitions \cite{wang_logic-based_2004, yuan_attributed_2005, cruz_constraint_2009}. Models for a variety of use cases have been proposed such as: an attribute-based access matrix \cite{zhang_attribute-based_2005}, \gls{abac} for webservices \cite{yuan_attributed_2005}, attribute-based encryption \cite{goyal_attribute-based_2006, wang_hierarchical_2010}, grid computing \cite{lang_flexible_2009}, \gls{abac} applications on \gls{rbac} \cite{kuhn_adding_2010}, cloud computing \cite{wan_hasbe:_2012, yang_attribute-based_2013}, \gls{iot} \cite{bhatt_access_2017, ouaddah_access_2017} and data sharing \cite{yu_attribute_2010}. There's been an attempt at unifying \gls{abac} into a model that could replace the previous models \cite{jin_unified_2012}, much like we have described for \gls{rbac}.

\subsection{Cryptography Access Control}

A research topic which aligns so deeply with our current digital environment finds early research in \citeyear{akl_cryptographic_1983}, with the proposal of a cryptographic solution for access control in organizations that are structured hierarchically \cite{akl_cryptographic_1983}. This research presents a system, which takes advantage of cryptographic keys, to allow a principal the derivation of cryptographic keys from each of the cryptographic keys below it, in a hierarchy. This ability allows an implemented system, in a structurally hierarchical organization, to be able to interpret security policies based solely on cryptographic methods.

Later, \citeauthor{mackinnon_optimal_1985} \cite{mackinnon_optimal_1985} suggest an optimal algorithm for the distribution of cryptographic keys, in the context of hierarchical access control. \citeauthor{sandhu_cryptographic_1988} \cite{sandhu_cryptographic_1988} proposes a solution that is different from what had been presented before in \cite{akl_cryptographic_1983}. \citeauthor{mackinnon_optimal_1985} \cite{mackinnon_optimal_1985}, \citeauthor{akl_cryptographic_1983} \cite{akl_cryptographic_1983} and \citeauthor{sandhu_cryptographic_1988}'s \cite{sandhu_cryptographic_1988} research was an evolution over \citeauthor{denning_master_1981}'s \cite{denning_master_1981} initial research, in the area of master cryptographic keys. This concept allowed the derivation of key cryptographic keys.

This historical basis is not only interesting as a subject of study, but also important for contextualizing the problem. The reason for this is that research initialized in the 1970s continues to this day, becoming increasingly relevant. From this research, carried out during the 1970s and the early 1980s, new research topics, and developments from the earlier ones, began to emerge. \citeauthor{brassard_flexible_1990} \cite{brassard_flexible_1990} presented a solution, with greater flexibility, to use master cryptographic keys in access control, evolving the initial works in \cite{akl_cryptographic_1983} and \cite{denning_master_1981}. \citeauthor{park_binding_2000} \cite{park_binding_2000} continue to develop their research applying digital certificates in access control. \citeauthor{miklau_controlling_2003} \cite{miklau_controlling_2003} demonstrate an interesting implementation that uses cryptography to control access to XML documents.

More recent research has also evolved the cryptographic use in access control, building on the blocks that have been proposed earlier \cite{di_vimercati_over-encryption:_2007}. Other works overlap with \gls{abac} \cite{wan_hasbe:_2012, ruj_privacy_2012, wang_hierarchical_2010, goyal_attribute-based_2006, harrington_cryptographic_2003}.

\subsection{Decentralized Access Control}

The study of access control in distributed systems has spanned several decades since the early developments of distributed systems through grid and cloud computing and cloud systems. Some of the early research developed in the field of access control in decentralized environments \cite{karger_non-discretionary_1977} encompassed much of the knowledge to date. As with initial research regarding access control models, \citeauthor{karger_non-discretionary_1977} \cite{karger_non-discretionary_1977} presents the challenges in distributed systems when applying MAC to a decentralized system. It recognizes that not all algorithms that guarantee access control in centralized systems can be easily mapped to a decentralized system, thus presenting a new way to work with lattice-based access control models (see Section \ref{sec:models-mac} that can be used in decentralized systems.

\citeauthor{moffett_specifying_1990} \cite{moffett_specifying_1990} proposed a specification for the implementation of discretionary
access control in distributed systems. In this research, the concept of domains, groups of objects and
access rules were introduced as a way of helping structure the way access control is managed in a highly
distributed and complex system. This work evolved from \citeauthor{satyanarayanan_integrating_1989} \cite{satyanarayanan_integrating_1989} on security in complex distributed systems. Research over the Andrew system \cite{satyanarayanan_integrating_1989} explored how to introduce access control in complex distributed systems.\citeauthor{sandhu_implementation_1992} \cite{sandhu_implementation_1992}, presented enhancements over  \citeauthor{sandhu_typed_1992}’s \cite{sandhu_typed_1992} Typed Access Matrix model. This research proposes a simpler TAM, called SO-TAM, that made it easier to implement a Typed Access Matrix, in distributed environments, as well as demonstrates the usage of digital certificates in access Control.

\citeauthor{johnston_authorization_1998} \cite{johnston_authorization_1998} present a mechanism for access control in distributed
systems. The system described leverages the research and capacity of digital certificates to allow access
control in distributed systems. \citeauthor{sandhu_decentralized_1998} \cite{sandhu_decentralized_1998} present the decentralization of access control management in RBAC models, in web-based systems. This research presents a way to try and decentralize the management of access control, but it still does not decentralize the policy engine behind the access control. It was an enhancement over a previous specification \cite{barkley_role_1997}. The research presented \cite{sandhu_decentralized_1998} evolved through the use of X.509 certificates \cite{park_smart_1999} and \citeauthor{park_role-based_2001} \cite{park_role-based_2001} return to examine this research.

\citeauthor{harrington_cryptographic_2003} \cite{harrington_cryptographic_2003} note that one of the main problems in the area of access control is the centralization of the reference monitors, and present a proposal for a cryptographic access control system, which is inherently distributed but not completely decentralized. This research evolves what has been presented earlier \cite{satyanarayanan_integrating_1989}. \citeauthor{park_role-based_2003} \cite{park_role-based_2003} demonstrate the application of the RBAC model in peer-to-peer environments, which is an evolution towards a more complete decentralization of access control. \citeauthor{abiteboul_electronic_2004} \cite{abiteboul_electronic_2004} present the implementation of a distributed system of access to confidential patient data, in a distributed way, through XML data that presents an evolution over the research produced earlier \cite{damiani_fine-grained_2002}.

\citeauthor{bhatti_x-gtrbac_2004} \cite{bhatti_x-gtrbac_2004} present a system - X-GTRBAC - for access control management, in GTRBAC model \cite{joshi_generalized_2005}, over decentralized environments. \citeauthor{chakraborty_trustbac:_2006} \cite{chakraborty_trustbac:_2006} present a new access control model, called TrustBAC \cite{chakraborty_trustbac:_2006}, to address a problem identified in RBAC: difficulty in assigning roles to subjects in decentralized systems, due to the fact that, in this type of systems, identities are often not known a priori and the “user population is dynamic”.

With the proliferation of decentralized systems, developments in the area of access control continue
to emerge. \citeauthor{ruj_dacc:_2011} \cite{ruj_dacc:_2011} present a new algorithm and model for the access control \cite{ruj_dacc:_2011} in clouds by expanding on the cryptographic research previously published. \citeauthor{calero_toward_2010} \cite{calero_toward_2010} propose a system for the same effect based on the RBAC model \cite{calero_toward_2010}. This model intends to describe a suitable architecture for an access control system for cloud computing. \citeauthor{yu_achieving_2010} \cite{yu_achieving_2010} present their model for access control in cloud computing, which is related to previous research \cite{calero_toward_2010}. Recently, a number of attribute-based access control models have been developed. \citeauthor{ruj_decentralized_2014} \cite{ruj_decentralized_2014} present research on access control in decentralized systems in which the identity of the principal is unknown \cite{ruj_decentralized_2014}. The research presented in \cite{ruj_decentralized_2014} extends what had already been done by \citeauthor{ruj_privacy_2012} \cite{ruj_privacy_2012}.

\section{Blockchain}
\label{sec:related-blockchain}

\subsection{Primer on Blockchains}

Blockchain technologies emerged as a supporting technology for the cryptocurrency Bitcoin, although under a different name \cite{nakamoto_bitcoin:_2008}. \citeauthor{nakamoto_bitcoin:_2008}'s contribution was to solve the \textit{double spending problem} without using a trusted central authority. For this, time stamping servers were used, which use the concept of blocks of hashes, with each hash having a reference to the previous block, and a modification of an existing proof-of-work algorithm \cite{back_hashcash_2002}, initially developed to mitigate Denial of Service attacks \cite[1]{back_hashcash_2002}. Consequently, the name for what we now call blockchains comes from the fact that, simply put, they can be represented as chains of blocks, with references to previous blocks. The concept proposed by \citeauthor{nakamoto_bitcoin:_2008} \cite{nakamoto_bitcoin:_2008} was then to have transactions of the cryptocurrency Bitcoin broadcast to all nodes on the network, after which each node would process that transaction and try to find a proof-of-work. When a node was successful, it would broadcast a new block to all other nodes, and that block would be accepted only if all previous transactions in it were valid and not spent before.

In essence, a blockchain can be perceived as a decentralized transaction ledger, with no central authority and no single point of failure. This ledger is maintained by a chain of blocks, that represent the transactions, and the creation of new blocks is managed through consensus' protocols. Until now, the concept of a blockchain doesn't have a formal definition but several definitions have been presented through new research. \citeauthor{buterin_next-generation_2013} \cite{buterin_next-generation_2013} suggests, appropriately, that one of the most revolutionizing aspects of the Bitcoin experiment is not the decentralized cryptocurrency, and the financial implications it might have, but rather the fact that this new blockchain concept can be \textit{"a tool of distributed consensus"} \cite{buterin_next-generation_2013} and presents \textit{Ethereum} as a platform for building decentralized applications, and not only cryptocurrencies. As predicted by \citeauthor{buterin_next-generation_2013} \cite{buterin_next-generation_2013}, and later described by \citeauthor{pilkington_blockchain_2016} \cite{pilkington_blockchain_2016}, blockchain technology, while still being at the very core of cryptocurrencies, started moving away and dwelling further into other applications. Blockchain's adoption started moving from, not only, Ethereum, or Ripple \cite{schwartz_ripple_2014}, towards more practical applications such as identity providers, voting systems, supply chain alternatives for enhanced transparency or banking applications \cite{pilkington_blockchain_2016}. At the same time other research has pointed towards different possible applications of blockchain technology, other than cryptocurrencies \cite{crosby_blockchain_2016, underwood_blockchain_2016, yermack_corporate_2017, xu_blockchain_2016}, which we will discuss, in more detail, in Section \ref{sec:related-blockchain}.

\subsection{Applications}

While the technology itself keeps evolving and under analysis \cite{eyal_bitcoin-ng:_2016, wang_research_2018, gervais_security_2016, lin_survey_2017}, and the entire concept of Bitcoin is already heavily based on previous academic literature \cite{narayanan_bitcoins_2017}, we are only now starting to apply this technology to practical problems of our society, albeit slowly and cautiosly. One of those issues, and one that has currently been the focus of major media attention, as well as in the academic environment, is the area of Information Security. Given its decentralizing nature, as well as its cryptographic background, blockchains have started to gather attention for their potential applications at the level of access control, privacy and security in general. This happens due to the increasingly distributed and federated environments in which we now live, such as the Internet of Things, which call for different approaches and concepts, when considering how to more effectively to secure them, as motivated in Section \ref{sec:intro-motivation}.

Research over using blockchain to resolve problems with \textit{data ownership} and privacy has been conducted in \cite{zyskind_decentralizing_2015}, \cite{liang_provchain:_2017} and \cite{yue_healthcare_2016}. It has also been researched as a way to enhance security and decentralization in content distribution \cite{fotiou_decentralized_2016}. At the same time, blockchain has also been researched as, potentially, an application for enhanced security of IoT infrastructure, in \cite{dorri_blockchain_2016}, \cite{dorri_blockchain_2017} and \cite{ouaddah_access_2017}. Furthermore, on the factor of access control and identity management, there's been some research developed by \citeauthor{augot_identity_2017} \cite{augot_identity_2017}, as well as \citeauthor{maesa_blockchain_2017} \cite{maesa_blockchain_2017}. In \cite{ouaddah_fairaccess:_2017}, an overlap between access control and IoT was researched, in which blockchain enhanced access control on IoT infrastructure. Finally, there's been research on applying blockchain technology into securing smart cities \cite{biswas_securing_2016}, decentralized private voting systems \cite{sheer_hardwick_e-voting_2018} and securing credit reporting \cite{kafshdar_goharshady_secure_2018}.

\section{Educational Certificates}
\label{sec:related-ec}

MIT's Media Lab Learning Initiative \cite{mit_learning_initiative}, along with Learning Machine \cite{learning_machine}, have conducted research, \textit{Digital Certificates Project} \cite{MITCertificates}, in 2015, on this subject. This research developed the initial prototype that allowed to create an ecosystem for issuing and sharing educational certificates, based on blockchains. Some certificates generated based on this prototype are still accessible \cite{MITCertificatesBootcamp}. \textit{Digital Certificates Project}, initially focused on issuing digital certifications for educational purposes, later spun \textit{BlockCerts} \cite{Blockcerts}, which expanded the issuing of certificates from educational certificates to more generic use cases. \textit{BlockCerts} claims to be \textit{"The Open Standard For Blockchain Credentials"} and, contrary to what had been developed for \textit{Digital Certificates Project}, it includes a more robust ecosystem, based on the open-sourced code of the initial prototype. \textit{BlockCerts} now includes libraries, apps and tools to allow the development of decentralized  applications for issuing and sharing digital (educational) certificates.

Although \textit{BlockCerts} is an effort to standardize the development of decentralized applications for these purposes, it currently lacks some functionality - some of it defined in its Roadmap - such as: \emph{(i)} only works with Bitcoin and Ethereum. This means there's a lack of support for additional public blockchains or federated blockchains; \emph{(ii)} revocation is highly dependent on the issuer and there's a lack of access control capabilities for the recipient of the certificate; \emph{(iii)} uses the same approach as cryptocurrencies - the wallet concept - to manage the issuing and sharing of the certificates.

As with any new technology, there's a lot of innovation space. \textit{BlockCerts} has a growing, strong open-source community and an ecosystem that enables some of the existing gaps to be closed. It also helps to state the relevance of the issue this thesis explores.

