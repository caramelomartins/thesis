\chapter{Related Work}
\label{chap:related}

In this chapter we present some relevant background and work that is related with this thesis. Due to the broad nature of our research problem, as described in the previous chapter, there's a need to overview several different areas of interest. We start by reviewing, in Section \ref{sec:related-ac}, existing literature on \gls{ac} models and applications, discussing how they related to this research and highlighting the chosen models for our use case. Section \ref{sec:related-blockchain} reviews different blockchain approaches, dissecting their components, and describes existing literature on blockchain applications. Section \ref{sec:related-crypto} presents a primer on cryptography systems, meant to be understood as background for the research presented, and discusses those systems related to this thesis. Finally, Section \ref{sec:related-ec} reviews some of the work done on the topic of Educational Certificates, which is the use case of focus in this thesis. By the end of this chapter, it should be possible to realize how these concepts, and existing literature, relate to the core contributions of thesis. It should also be possible to contextualize our core research within the existing literature and, at the same time, lay out the necessary concepts to understand the forthcoming chapters.

\section{Access Control}
\label{sec:related-ac}

\glsdesc{ac} has been a core component in every technology evolution and continues to be as relevant today. It has always played a central role in technology because information is a valuable asset that must be protected from prying hands. Due to that fact, both governments and corporations have spent resources on developing \gls{ac} models and instantiations, while academia produced a broad spectrum of literature on the topic. Said literature has been produced with all sorts of approaches, from the purely theoretical to implementation-based approaches. Nonetheless, no matter what has been produced and developed previously, there are always new challenges to be faced and, with every new evolution, a new challenge emerges, which has led the field to be one of extensive study ever since the beginnings of the computer age. As more sensitive data is shared and stored by third-parties, the relevance of \gls{ac} is only increasing, rather than decreasing, creating an optimal context for researchers to be able to developed interesting new research that solves important problems.

\subsection{Approaches and Models}

\subsubsection{\gls{mac}}

\glsdesc{mac} is a keystone \gls{ac} approach characterized by embedding access control policies inside systems rather than in external sources. To help clarify, let us imagine an application, written in a given programming language, which has access to a given set of resources. Using \gls{mac}, the access control policies for a subject to access each of the resources would have to be embedded - hard-coded through programming - into the application itself such that any adjustment to the access control policies would require an adjustment to the application itself. This approach emerged in the early stages of \gls{ac} research, when research on this topic was very focused for military applications.

\subsubsection{\gls{dac}}

\subsubsection{\gls{rbac}}

\subsubsection{\gls{abac}}

\subsubsection{Cryptography Access Control}

\subsection{Applications}

\subsection{Discussion}

\section{Blockchain}
\label{sec:related-blockchain}

\subsection{Public vs. Private}

\subsection{Permissionless vs. Permissioned}

\subsection{Applications}

BlockChain technologies emerged as a supporting technology for the cryptocurrency Bitcoin, although under a different name \cite{nakamoto_bitcoin:_2008}. \citeauthor{nakamoto_bitcoin:_2008}'s contribution was to solve the \textit{double spending problem} without using a trusted central authority. For this, time stamping servers were used, which use the concept of blocks of hashes, with each hash having a reference to the previous block, and a modification of an existing proof-of-work algorithm \cite{back_hashcash_2002}, initially developed to mitigate Denial of Service attacks. Thus, the name for what we now call BlockChains comes from the fact that, simply put, they can be represented as chains of blocks, with references to previous blocks. The concept proposed by \citeauthor{nakamoto_bitcoin:_2008} was then to have transactions of the cryptocurrency Bitcoin broadcast to all nodes on the network, after which each node would process that transaction and try to find a proof-of-work. When a node was successful, it would broadcast its block to all other nodes, and that block would be accepted only if all previous transactions in it were valid and not spent before.

In essence, a BlockChain can be perceived as a decentralized transaction ledger, with no central authority and no single point of failure. This ledger is maintained by a chain of blocks, that represent the transactions, and the creation of new blocks is managed through consensus' protocols. Until now, the concept of a BlockChain doesn't have a formal definition but several definitions have been presented through new research. In \cite{buterin_next-generation_2013}, \citeauthor{buterin_next-generation_2013} suggests, appropriately, that one of the most revolutionizing aspects of the Bitcoin experiment is not the decentralized cryptocurrency, and the financial implications it might have, but rather the fact that this new BlockChain concept can be \textit{"a tool of distributed consensus"} and presents \textit{Ethereum} as a platform for building decentralized applications, and not only cryptocurrencies. As predicted by \citeauthor{buterin_next-generation_2013}, and later described by \citeauthor{pilkington_blockchain_2016}, in \cite{pilkington_blockchain_2016}, BlockChain technology, while still being at the very core of cryptocurrencies, started moving away and dwelling further into other applications. In \cite{pilkington_blockchain_2016} we can BlockChains adoption started moving from, not only, Ethereum, or Ripple, towards more practical applications such as identity providers, voting systems, supply chain alternatives for enhanced transparency or banking applications. At the same time other research has pointed towards different possible applications of BlockChain technology, other than cryptocurrencies (\cite{crosby_blockchain_2016}, \cite{underwood_blockchain_2016}, \cite{yermack_corporate_2017}, \cite{xu_blockchain_2016}).

While the technology itself keeps evolving and under analysis (\cite{eyal_bitcoin-ng:_2016}, \cite{wang_research_2018}, \cite{gervais_security_2016}, \cite{lin_survey_2017}), and the entire concept of Bitcoin is already heavily based on previous academic literature \cite{narayanan_bitcoins_2017}, we are only now starting to apply this technology to practical problems of our society. One of those issues, and one that has currently been the focus of major media attention, as well as in the academic environment, is the area of Information Security. Given its decentralizing nature, as well as its cryptographic background, BlockChains have started to gather attention for their potential applications at the level of access control, privacy and security in general. This happens due to the increasingly distributed and federated environments in which we now live, such as the Internet of Things, which call for different approaches and concepts, when considering how to more effectively to secure them.

Research over using BlockChain to resolve problems with \textit{data ownership} and privacy has been conducted in \cite{zyskind_decentralizing_2015}, \cite{liang_provchain:_2017} and \cite{yue_healthcare_2016}. It has also been researched as a way to enhance security and decentralization in content distribution \cite{fotiou_decentralized_2016}. At the same time, BlockChain has also been researched as, potentially, an application for enhanced security of IoT infrastructure, in \cite{dorri_blockchain_2016}, \cite{dorri_blockchain_2017} and \cite{ouaddah_access_2017}. Furthermore, on the factor of access control and identity management, there's been some research developed by \citeauthor{augot_identity_2017} in \cite{augot_identity_2017}, as well as \citeauthor{maesa_blockchain_2017} in \cite{maesa_blockchain_2017}. In \cite{ouaddah_fairaccess:_2017}, an overlap between access control and IoT was researched, in which BlockChain enhanced access control on IoT infrastructure. Finally, there's been research on applying BlockChain technology into securing smart cities \cite{biswas_securing_2016}, decentralized private voting systems \cite{sheer_hardwick_e-voting_2018} and securing credit reporting \cite{kafshdar_goharshady_secure_2018}.

\subsection{Discussion}

\section{Cryptography}
\label{sec:related-crypto}

\subsection{Symmetric Encryption}

\subsection{Assymetric Encryption}

\subsection{Discussion}

\section{Educational Certificates}
\label{sec:related-ec}

MIT's Media Lab Learning Initiative, along with Learning Machine, have conducted research, \textit{Digital Certificates Project} \cite{MITCertificates}, in 2015, on this subject. This research developed the first prototype, to the authors' knowledge, that allowed to create an ecosystem for issuing and sharing educational certificates, based on BlockChains. Some certificates generated based on this prototype are still accessible \cite{MITCertificatesBootcamp}. \textit{Digital Certificates Project}, initially focused on issuing digital certifications for educational purposes, later spun \textit{BlockCerts} \cite{Blockcerts}, which expanded the issuing of certificates from educational certificates to more generic use cases. \textit{BlockCerts} claims to be \textit{"The Open Standard For BlockChain Credentials"} and, contrary to what had been developed for \textit{Digital Certificates Project}, it includes a more robust ecosystem, based on the open-sourced code of the initial prototype. \textit{BlockCerts} now includes libraries, apps and tools to allow the development of decentralized  applications for issuing and sharing digital (educational) certificates.

Although \textit{BlockCerts} is an effort to standardize the development of decentralized applications for these purposes, it currently lacks some functionality - some of it defined in its Roadmap - such as: \emph{(i)} only works with Bitcoin and Ethereum. This means there's a lack of support for additional public BlockChains or federated BlockChains; \emph{(ii)} revocation is highly dependent on the issuer and there's a lack of access control capabilities for the recipient of the certificate; \emph{(iii)} uses the same approach as cryptocurrencies - the wallet concept - to manage the issuing and sharing of the certificates.

As with any new technology, there's a lot of innovation space. \textit{BlockCerts} has growing a strong open-source community and an ecosystem that enables some existing gaps to be closed. It also helps to state the relevance of the issue at hand, presented in this section.

\section{Summary}

